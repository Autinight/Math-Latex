\documentclass[lang=cn,12pt,color=blue,pad,fontset=none]{elegantbook}
%fontset=none,
\title{几何与拓扑}
% \subtitle{课内复习版}

\author{Autin}
\usepackage{stmaryrd}

\usepackage{quiver} 

\setmainfont{Aa顺風顺水顺财神}[BoldFont=Shanggu Round-Bold]
\setCJKmainfont{Aa顺風顺水顺财神}[BoldFont=Shanggu Round-Bold]
\setCJKsansfont{Aa顺風顺水顺财神}[BoldFont=Shanggu Round-Bold]
\setCJKmonofont{Aa顺風顺水顺财神}[BoldFont=Shanggu Round-Bold]
 
\usepackage{subfiles}
\extrainfo{} 

\setcounter{tocdepth}{3}

% \logo{xmu-logo.pdf}
\cover{image.png}
\usepackage{CJKutf8}
% 本文档命令
\usepackage{array}
\newcommand{\ccr}[1]{\makecell{{\color{#1}\rule{1cm}{1cm}}}}
 
% 修改标题页的橙色带
% \definecolor{customcolor}{RGB}{32,178,170}
% \colorlet{coverlinecolor}{customcolor}

\begin{document}

\maketitle

\frontmatter

\tableofcontents

\mainmatter

\part{点集拓扑}

\subfile{chapters/拓扑空间/拓扑空间.tex}

\part{微分流形}

\subfile{chapters/光滑流形/光滑流形.tex}

\subfile{chapters/子流形/子流形.tex}

\subfile{chapters/李群/李群.tex}

\subfile{chapters/向量场/向量场.tex}

\subfile{chapters/张量/张量.tex}

\subfile{chapters/微分形式/微分形式.tex}

\subfile{chapters/de Rham上同调/de Rham上同调.tex}

\part{黎曼流形}

\subfile{chapters/黎曼度量/黎曼度量.tex}

\subfile{chapters/模型黎曼流形/模型Riemann流形.tex}

\subfile{chapters/联络/联络.tex}

\subfile{problems/联络/联络.tex}

\subfile{chapters/Levi-Civita联络/Levi-Civita联络.tex}

\subfile{chapters/测地线和距离/测地线和距离.tex}

\subfile{chapters/曲率/曲率.tex}

\subfile{problems/曲率/曲率.tex}

\subfile{chapters/黎曼子流形/黎曼子流形.tex}

\subfile{problems/黎曼子流形/黎曼子流形.tex}
                                    
\subfile{chapters/Gauss-Bonnet定理/Gauss-Bonnet定理.tex}

\subfile{chapters/Jacobi场/Jacobi场.tex}

\part{代数拓扑}

\subfile{chapters/范畴论基础/范畴论基础.tex}

\subfile{chapters/胞腔复形/胞腔复形.tex}

\subfile{chapters/基本群/基本群.tex}

\subfile{chapters/基础同调代数/基础同调代数.tex}

\subfile{chapters/奇异同调/奇异同调.tex}

\subfile{problems/奇异同调/奇异同调.tex}

\subfile{chapters/覆叠空间/覆叠空间.tex}

\subfile{chapters/单纯复形/单纯复形.tex}

\subfile{chapters/单纯同调/单纯同调.tex}

\subfile{chapters/知识点总结/知识点总结.tex}
\end{document}
 

 