\documentclass[../../几何与拓扑.tex]{subfiles}

\begin{document}
 
\ifSubfilesClassLoaded{
    \frontmatter

    \tableofcontents
    
    \mainmatter
}{}

\chapter{知识点总结}
\begin{tabular}{|l | c | c | c|}
\hline
性质组合             & $H^n(M; \mathbb{Z})$ & $H^n(M; R)$ (\(  R  \)  是域, $R \neq \mathbb{Z}_2$) & $H^n(M; \mathbb{Z}_2)$ \\
\hline
紧致且可定向         & $\mathbb{Z}$         & $R$                                       & $\mathbb{Z}_2$         \\
\hline
紧致且不可定向       & $0$                  & $0$                                       & $\mathbb{Z}_2$         \\
\hline
非紧致且可定向       & $0$                  & $0$                                       & $0$                    \\
\hline
非紧致且不可定向     & $0$                  & $0$                                       & $0$                    \\
\hline
\end{tabular}
\begin{theorem}{Kunneth公式}
    对于同调
   $$
H_k(X \times Y; R) \cong \left( \bigoplus_{i+j=k} (H_i(X; R) \otimes_R H_j(Y; R)) \right) \oplus \left( \bigoplus_{i+j=k-1} \text{Tor}_1^R(H_i(X; R), H_j(Y; R)) \right)
$$
对于上同调
$$
H^k(X \times Y; R) \cong \left( \bigoplus_{i+j=k} (H^i(X; R) \otimes_R H^j(Y; R)) \right) \oplus \left( \bigoplus_{i+j=k+1} \text{Ext}_R^1(H^i(X; R), H^j(Y; R)) \right)
$$
\end{theorem}

特别地,域上的Kunneth公式有简洁的形式

\begin{theorem}{域上的Kunneth公式}
  设 \(  F  \)上一个域
  
  对于同调$$
H_k(X \times Y; F) \cong \bigoplus_{i+j=k} (H_i(X; F) \otimes_F H_j(Y; F))
$$对于上同调$$
H^k(X \times Y; F) \cong \bigoplus_{i+j=k} (H^i(X; F) \otimes_F H^j(Y; F))
$$
\end{theorem}
\begin{remark}
   \begin{enumerate}
    \item  由此导出维数的计算公式,从而可以方便地计算Betti数,Euler示性式.
    \item 在张量积环中,乘积的定义为 \[
    \left( \alpha _1 \otimes \beta _1  \right)\cdot \left( \alpha _2 \otimes \beta _2  \right) =  \left( -1 \right)^{\left| \beta _1  \right|\left| \alpha _2  \right|  }\left( \alpha _1 \cup \alpha _2  \right)\otimes \left( \beta _1 \otimes \beta _2  \right)     
    \]
   \end{enumerate}
   
\end{remark}
\end{document}