\documentclass[../../几何与拓扑.tex]{subfiles}

\begin{document}

\begin{problemsec}
    
    \begin{problem}
        \begin{enumerate}
            \item 证明\(  H_0\left( X,A \right)  = 0\),当且仅当 \(  A  \)达到 \(  X  \)的每个道路分支.   
        \end{enumerate}
        
    \end{problem}
    \begin{proof}
        \(  A  \)达到 \(  X  \)的每个道路分支,当且仅当对于任意的 \(  x \in X  \),都存在 \(  a \in A  \),使得 \(  x,a  \)成为 \(  X  \)上一个道路的端点.由于道路上奇异1-单形,这相当说,当视 \(  x  \)为 \(  X  \)上的0-单形, \(  a  \)为 \(  A  \)上的0-单形时, \(  x -i_{\sharp }\left( a \right)\in \operatorname{Im}\, \partial _{1}   \),即同调类 \(  [x]= [i_{\sharp }\left( a \right) ]= i_{*}\left( [a] \right)   \)     ,其中 \(  i: A\hookrightarrow X  \)是含入映射.断言上述成立,当且仅当 \(  i_{*}  \)是满射: 由于 \(  X,A  \)上的点与它们上的0-单形的一组基对应,由于 \(  i_{*}  \)线性,\(  X  \)上的任意0-单形写成 这这组 \(  X  \)上0-单形 基的线性组合,进而这组 \(  A  \)的 \(  0  \)-单形基的线性组合的含入像,这表明 \(  i_{*}  \)是满射. 反之若 \(  i_{*}  \)是满射,   \(  x \in X  \),存在 \(  a =  \sum m_{k}a_{k}  \),其中 \(  a_{k}\in A\subseteq X  \), 使得 \(  [x]=i_{*}\left( [a] \right) = \sum m_{k}i_*\left( [a_{k}] \right) \).   这表明 \(  x- \sum m_{k}a_{k}  \)上的一个闭链,故 \(  \sum m_{k}= 1  \).  
        
        另一方面,根据 \(  X,A  \)相对同调的长正合列, \(  H_0\left( X,A \right)   = 0\)当且仅当 \(  i_{*}: H_0\left( A \right)\to H_0\left( X \right)    \)是满射,   这当且仅当对于任意的 \(  [x] \in H_0\left( X \right)   \),存在 \(  [a ] \in H_0\left( A \right)   \),使得 \(  [x]-i_{*}\left( [a] \right)= 0   \).
        \hfill $\square$
    \end{proof}
    \hspace*{\fill} 
    
\end{problemsec}

\end{document}