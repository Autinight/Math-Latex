\documentclass[lang=cn,12pt,color=green,fontset=none]{elegantbook}

\title{标题}

\author{Autin}

\setmainfont{Aa顺風顺水顺财神}
\setCJKmainfont{Aa顺風顺水顺财神}
\setCJKsansfont{Aa顺風顺水顺财神}
\setCJKmonofont{Aa顺風顺水顺财神}

%\extrainfo{不要以为抹消过去,重新来过,即可发生什么改变.—— 比企谷八幡}

\setcounter{tocdepth}{3}


\cover{image.png}
\usepackage{CJKutf8}
% 本文档命令
\usepackage{array}
\newcommand{\ccr}[1]{\makecell{{\color{#1}\rule{1cm}{1cm}}}}

% 修改标题页的橙色带
% \definecolor{customcolor}{RGB}{32,178,170}
% \colorlet{coverlinecolor}{customcolor}

\begin{document}
\large
\maketitle
\frontmatter

\tableofcontents

\mainmatter
\chapter{同余式 }

\section{一次同余式}

\begin{theorem}
    \[
    ax\equiv b\left( \operatorname{mod}\, m \right),\quad a \not\equiv  0 \left( \operatorname{mod}\,m \right)  
    \]当且仅当  \(  \left( a,m \right) |b   \) ,此时解数为 \(  d =  \left( a,m \right)   \) 
\end{theorem}

\noindent 解一次同余式的算法:
\begin{enumerate}
    \item 找最大公因子 \(  d =  \left( a,m \right)   \);
    \item \(  a,b,m  \)除以 \(  d  \),化作系数和模数互素的情况. \[
    a_1x \equiv b_1 \left( \operatorname{mod}\,m_1 \right) ,\quad \left( a_1,m_1 \right)= 1 
    \]   
    \item 找到同余式 \[
    a_1x \equiv 1\left( \operatorname{mod}\,m_1 \right) 
    \]的解 \(  c  \).
    \begin{enumerate}
        \item 矩阵行变换形式的辗转相除法;
        \item 欧拉定理 \[
        c =  a_1^{\varphi \left( m_1 \right)-1 }
        \]
    \end{enumerate}
     \item 两边乘以 \(  c  \),得到解 \[
     x \equiv b_1c \left( \operatorname{mod}\,m_1 \right) 
     \] 为模 \(  m_1  \)意义下的唯一解.
     \item 所有解为 \[
     b_1c+  k\cdot m_1\left( \operatorname{mod}\,m \right),\quad k= 0,1,\cdots ,d-1 
     \] 
\end{enumerate}
\section{孙子定理}

\begin{theorem}{CRT}
    设 \(   m_1,\cdots,m_k   \)是 \(  k  \)个两两互素的正整数,令 \[
    m=  m_1\cdots m_{k},\quad m= m_{i}M_{i},\quad i=  1,\cdots,k ,
    \]则同余式组 \[
    \begin{cases} x \equiv b_1&\left( \operatorname{mod}\,m_1 \right)\\ 
     \vdots\\ 
      x \equiv b_{k}&\left( \operatorname{mod}\,m_{k}  \right) \end{cases} 
    \]  在 \(  \operatorname{mod}\,m  \)意义下的唯一解是 \[
    x \equiv M_1^{\prime} M_1b_1+ \cdots + M^{\prime} _{k}M_{k}b_{k}\;\left( \operatorname{mod}\,m \right) 
    \] 其中 \[
    M_{i}^{\prime} M_{i}\equiv 1 \;\left( \operatorname{mod}\, m_{i} \right),i=  1,\cdots,k  
    \]
\end{theorem}

\begin{theorem}
    若 \(   b_1,\cdots,b_k   \)过模 \(   m_1,\cdots,m_k   \)的完全剩余系,则 \[
    M_{1}^{\prime} M_1b_1+ \cdots + M_{k}^{\prime} M_{k}b_{k}
    \]  过模 \(  m  \)的完全剩余系. 
\end{theorem}
\begin{remark}
    给出寻找合数模的完全剩余系的方法,只需要做素因子分解,给出每个准素数的完全剩余系,并找出系数 \(  M_{i}^{\prime} M_{i}  \),拼成模 \(  m  \)的完全剩余系.  
\end{remark}

\section{高次同余式}

\begin{theorem}{合数模变为准素模}
    若 \(   m_1,\cdots,m_k   \)是 \(  k  \)个两两互素的正整数,令 \(  m = m_1m_2\cdots m_{k}  \),则 \[
    f\left( x \right)\equiv 0\left( \operatorname{mod}\,m \right)\iff  f\left( x \right)\equiv 0\left( \operatorname{mod}\,m_{i} \right), i=  1,\cdots,k     
    \]  模 \(  m  \)解的个数为模 \(  m_{i}  \)解的个数的乘积.   
\end{theorem}

\begin{theorem}{准素模变为素数模}
    设 \(  p  \)是素数.若 \(  x \equiv x_1\left( \operatorname{mod}\,p \right)   \)是 \(  f\left( x \right)\equiv 0\left( \operatorname{mod}\,p \right)    \)的一个解,并且 \(  p \nmid f^{\prime} \left( x_1 \right)   \),则存在 \(  \operatorname{mod}\, p^{ \alpha }  \) 下唯一的 \(  x_{\alpha }  \),
    使得 \(  x_{\alpha }\equiv x_1\left( \operatorname{mod}\,p \right)   \),且 \(  x\equiv x_{\alpha }  \left( \operatorname{mod}\,p^{\alpha } \right) \)成为 \(  f\left( x \right)\equiv 0\left( \operatorname{mod}\,p^{\alpha } \right)    \)的一个解.   
\end{theorem}

\noindent 准素数模同余式的算法:
对于同余式 \(  f\left( x \right)\equiv 0\left( \operatorname{mod}\,p^{\alpha } \right)    \) 
\begin{enumerate}
    \item 求同余式 \[
    f\left( x \right)\equiv 0\left( \operatorname{mod}\, p \right)  
    \]的一个解 \(  x_1  \),使得 \(  p \nmid f^{\prime} \left( x_1 \right)   \)  
    \item 寻找 \(  t_1  \),使得 \[
    f\left( x_1+ pt_1 \right)\equiv 0\left( \operatorname{mod}\,p^{2} \right)  
    \] 为此,将左式泰勒展开 ,得到 \[
    f\left( x_1 \right)+ pt_1f^{\prime} \left( x_1 \right)\equiv 0 \left( \operatorname{mod}\,p^{2} \right)   
    \]解一次同余式,得到唯一解 \(  t_1\equiv t_1^{\prime} \left( \operatorname{mod}\,p^{2} \right)   \). 

    令 \(  x_2: =  x_1+ pt_1^{\prime}   \),则 \(  x \equiv x_2\left( \operatorname{mod}\,p^{2} \right)   \)是 \(  f\left( x \right)\equiv 0\left( \operatorname{mod}\,p^{2} \right)    \)的一个解.   

    \item 重复上述操作,对于以上过程中得到的 \(  f\left( x \right)\equiv 0\left( \operatorname{mod}\,p^{\alpha -1} \right)    \)的解 \(  x \equiv x_{\alpha -1}  \left( \operatorname{mod}\,p^{\alpha -1} \right) \)  .
    
    寻找 \(  t_{\alpha -1}  \),使得 \[
    f\left( x_{\alpha -1}+ p^{ \alpha -1}t_{ \alpha -1} \right)\equiv 0\left( \operatorname{mod}\,p^{ \alpha } \right)  
    \] 为此,Taylor展开得到 \[
    f\left( x_{\alpha -1} \right)+ p^{ \alpha -1}t_{ \alpha -1}f^{\prime} \left( x_{ \alpha -1} \right)  \equiv  0\left( \operatorname{mod}\,p^{ \alpha } \right) 
    \]解一次同余式,得到唯一解 \(  t_{ \alpha -1}\equiv t_{ \alpha -1}^{\prime} \left( \operatorname{mod}\,p^{ \alpha } \right)   \). 
    
    令\(  x_{ \alpha }: =  x_{ \alpha -1}+ t ^{\prime} _{ \alpha -1} p^{ \alpha -1}  \) ,则 \(  x \equiv x_{ \alpha }\left( \operatorname{mod}\,p^{ \alpha } \right)   \)为同余式 \(  f\left( x \right)\equiv 0   \left( \operatorname{mod}\,p^{ \alpha } \right) \)的一个解.
 
\end{enumerate}
\section{素数模的同余式}
考虑素数模 \(  p  \)的同余式 \[
f\left( x \right)\equiv 0\left( \operatorname{mod}\,p \right),\quad f\left( x \right)= a_{n}x^{n}+ a_{n-1}x^{n-1}+ \cdots + a_0   
\] 其中 \(  p  \)是素数, \(  a_{n } \not \equiv 0\left( \operatorname{mod}\,p \right)   \)  
\begin{theorem}
    上述同余式与一个次数不超过 \(  p-1  \)的模 \(  p  \)的同余式等价.  
\end{theorem}

\begin{theorem}
    设 \(  k\le n  \), \(  x \equiv  \alpha _{i}\left( \operatorname{mod}\,p \right)   \)  \(  \left(  1,\cdots,k  \right)   \)是素数模同余式的 \(  k  \)个不同的解,则对于任意的整数 \(  x  \),我们有 \[
    f\left( x \right)\equiv \left( x- \alpha _1  \right)\left( x- \alpha _2  \right)\cdots \left( x- \alpha _k  \right)    f_{k}\left( x \right)\left( \operatorname{mod}\,p \right)  
    \]   其中 \(  f_{k}  \)是首项系数为 \(  a_{n}  \)的 \(  n-k  \)次多项式.   
\end{theorem}

\begin{theorem}
    \begin{enumerate}
        \item  \[
        x^{p-1}-1 \equiv \left( x-1 \right)\left( x-2 \right)\cdots \left( x-\left( p-1 \right)  \right) \left( \operatorname{mod}\,p \right)    
        \]
        \item \[
        \left( p-1 \right)!+ 1\equiv 0\left( \operatorname{mod}\,p \right)  
        \]
    \end{enumerate}
    
\end{theorem}

\begin{theorem}
    若 \(  n\le p  \),同余式 \[
    f\left( x \right)\equiv 0\left( \operatorname{mod}\,p \right),\quad f\left( x \right)= x^{n}+ a_{n-1}x^{n-1}+ \cdots + a_0   
    \]有\(  n  \)个解当且仅当 \(  f\left( x \right)   \)除 \(  x^{p}-x  \)所得余式的一切系数都是 \(  p  \)的倍数.        
\end{theorem}
\chapter{二次同余式}

\section{一般二次同余式的化简}

\begin{definition}
    二次同余式是指 \[
    ax^{2}+ bx+ c \equiv 0\left( \operatorname{mod}\,m \right) ,\quad a \not\equiv  0\left( \operatorname{mod}\,m \right) 
    \]
\end{definition}

需要讨论二次同余式什么时候有解.

第一步,将 \(  m  \)标准分解,化简为每个准素数模的同余式是否有解的问题
\begin{theorem}
    设 \(  m  \)的标准分解是 \(  m =  p_1^{ \alpha_1 }p_2^{ \alpha_2 }\cdots p_{k}^{ \alpha _k }  \),则上述二次同余式有解,当且仅当下列每个同余式都有解 \[
        ax^{2}+ bx+ c\equiv 0 \left( \operatorname{mod}\,p_{i}^{ \alpha _{i}} \right),\quad  i=  1,\cdots,k  
        \]  
\end{theorem}

接下来讨论准素数模同余式何时有解,

\begin{theorem}
    对于二次同余式 \[
    f\left( x \right) \equiv 0\left( \operatorname{mod}\,p^{ \alpha } \right),\quad f\left( x \right)= ax^{2}+ bx+ c   
    \]

    \begin{enumerate}
        \item 当 \(  p^{\alpha } | \left( a,b,c \right)   \)时,任意整数满足同余式,进而有解;
        \item 若 \(  p^{\alpha }  \)不整除 \(  \left( a,b,c \right)   \),不妨只考虑 \(  p\nmid(a,b,c)  \)的情况\footnote{可以让 \(  \left( a,b,c \right)   \)  除尽 \(  p  \),直到 \(  p \nmid \left( a,b,c \right)   \),化简为新的形如上的二次同余式.}   
        \begin{enumerate}
            \item 若 \(  p\mid a,p\mid b, p \nmid c  \),无解.
            \item  若 \(  p\mid a, p\nmid b  \),  
        \end{enumerate}
        
    \end{enumerate}
    
\end{theorem}

\section{奇素数的平方剩余与非剩余}
只讨论奇素数 \(  p  \)的平方剩余与非剩余,即讨论 \[
x^{2}\equiv a \left( \operatorname{mod}\,p \right),\quad \left( a,p \right)= 1  
\]的同余式的解.  
\begin{theorem}{欧拉判别}
    若 \(  \left( a,p \right)   = 1\), \(  p  \)是奇素数, 则 \(  a  \)是模 \(  p  \)的平方剩余,当且仅当 \[
    a^{\frac{p-1 }{2 } }\equiv 1\left( \operatorname{mod}\,p \right) 
    \]非剩余当且仅当 \[
    a^{\frac{p-1 }{2 } }\equiv -1\left( \operatorname{mod}\,p \right) 
    \]    若为平方剩余,则 \(  x^{2}\equiv a\left( \operatorname{mod}\,p \right)   \)恰有二解. 
\end{theorem}

\begin{theorem}
    模 \(  p  \)的既约剩余系有 \(  p-1  \)(偶数)个,  其中平方剩余与非剩余各占一半,有 \(  \frac{p-1 }{ 2}   \)个.
    
    其中的平方剩余在同余的意义下与 \[
    1^{2},2^{2},\cdots , \left( \frac{p-1 }{2 }  \right)^{2} 
    \]一一对应.
\end{theorem}

\section{利用勒让德符号判定}

\begin{theorem}{勒让德符号}
    设 \(  p  \)是奇素数,定义勒让德符号 \(  \left( \frac{a }{p }  \right)   \)按 \[
    \left( \frac{a }{p }  \right): =  \begin{cases} 1,& a\text{是模 p的平方剩余}\\ 
    -1,& a \text{是模p的平方非剩余},\\ 
0,& p|a \end{cases}  
    \]
\end{theorem}

\begin{proposition}{勒让德符号的运算性质}
    \begin{enumerate}
        \item 若 \(  a \equiv a_1\left( \operatorname{mod}\,p \right)   \),则 \[
        \left( \frac{a }{p }  \right) =  \left( \frac{a_1 }{p }  \right)  
        \] 
        \item  \[
        \left( \frac{a }{p }  \right)\equiv a^{\frac{p-1 }{2 }  }\left( \operatorname{mod}\,p \right)  
        \]
        \item \[
            \left(\frac{a_1a_2\cdots a_n}{p}\right)=\left(\frac{a_1}{p}\right)\left(\frac{a_2}{p}\right)\cdots\left(\frac{a_n}{p}\right)
        \]
        \item \[
        \left( \frac{ a b^{2} }{ p}  \right) =  \left( \frac{a }{p }     \right),\quad p\nmid b  
        \]
        \item \[
        \left( \frac{2 }{p }  \right) =  \left( -1 \right)^{\frac{p^{2}-1 }{ 8} }  
        \]
        \item 若 \(  \left( a,p \right)= 1   \)且 \(  2\nmid a  \),则 \[
        \left( \frac{a }{p }     \right) =  \left( -1 \right)^{ \sum _{k= 1}^{p_1} [\frac{ak }{p } ]},\quad  p_1 =  \frac{p-1 }{2 }   
        \]  
        \item  若 \(  p,q  \)是奇素数, \(  \left( p,q \right)= 1   \),则 \[
        \left( \frac{q }{p }  \right) =  \left( -1 \right)^{ \frac{p-1 }{2 }\cdot \frac{q-1 }{2 }  }  \left( \frac{p }{q }  \right) 
        \]  
    \end{enumerate}

\end{proposition}

\section{雅可比符号} 
引入以下雅可比符号,可以更方便地计算勒让德符号
\begin{definition}
    对于奇数 \(  m  \),定义雅可比符号 \(  \left( \frac{a }{m }  \right)   \),按 \[
    \left( \frac{a }{m }  \right): = \left( \frac{a }{p_1 }  \right)\left( \frac{a }{p_2 }  \right)\cdots \left( \frac{a }{p_{r} }  \right)    
    \]  其中 \(  m= p_1p_2\cdots p_{r}  \),\(  p_{i}  \)  是素数, \(  \left( \frac{a }{p_{i} }  \right)   \)是勒让德符号. 
\end{definition}

\begin{proposition}
    \begin{enumerate}
        \item 若 \(  a\equiv a_1\left( \operatorname{mod}\,m \right)   \),则 \[
        \left( \frac{a }{m }  \right) =  \left( \frac{a_1 }{m }  \right)  
        \] 
        \item  \[
        \left( \frac{-1 }{m}  \right)\equiv {-1}^{\frac{m-1 }{2 }  }\left( \operatorname{mod}\,m\right)  
        \]
        \item \[
            \left(\frac{a_1a_2\cdots a_n}{m}\right)=\left(\frac{a_1}{m}\right)\left(\frac{a_2}{m}\right)\cdots\left(\frac{a_n}{m}\right)
        \]
        \item \[
        \left( \frac{ a b^{2} }{ m}  \right) =  \left( \frac{a }{m }     \right),\quad \left( b,m \right)= 1  
        \]
        
        \item \[
        \left( \frac{2 }{m }  \right) =  \left( -1 \right)^{\frac{m^{2}-1 }{ 8} }  
        \]
       
        \item  若 \(  m,n  \) 是大于1的奇数,则 \[
        \left( \frac{n   }{m }  \right) =  \left( -1 \right)^{ \frac{m-1 }{2 }\cdot \frac{n-1 }{2 }  }  \left( \frac{m }{n }  \right) 
        \]  
    \end{enumerate}
\end{proposition}

\begin{exercise}
    判断同余式 \[
    x^{2}\equiv 286\left( \operatorname{mod}\,563 \right) 
    \]是否有解.    
\end{exercise}
\begin{solution}
     \[
     \begin{aligned}
     \left( \frac{286 }{563 }  \right)&  =    \left( \frac{2 }{563 }  \right)\left( \frac{143 }{563 }  \right)  \\ 
      & =  \left( -1 \right) \left( -1 \right)^{\frac{143-1 }{ 2} \cdot \frac{563-1 }{2 } } \left( \frac{563 }{143 }  \right)  \\ 
       & =  \left( \frac{-9 }{143 }  \right) =  \left( \frac{-1 }{143 }  \right)  =   
     \end{aligned}
     \]
\end{solution}

\chapter{同余}


\section{欧拉定理与费马小定理}

\begin{theorem}{欧拉 }

    设 \(  m  \)是大于 \(  1  \)的整数, \(  \left( a,m \right)= 1   \),则 \[
    a^{\varphi \left( m \right) }\equiv 1\left( \operatorname{mod}\,m \right) 
    \]   
\end{theorem}

\begin{theorem}{费马小定理}
    若 \(  p  \)是素数,则 \[
    a^{p}\equiv a\left( \operatorname{mod}\,p \right) 
    \] 
\end{theorem}

\chapter{原根和指标}

\section{指数及其基本性质}

\begin{definition}
    若 \(  m>1,\left( a,m \right)= 1   \),则使得同余式 \[
    a^{ \gamma } \equiv 1\left( \operatorname{mod}\,m \right) 
    \]成立的最小正整数 \(   \gamma   \)叫做 \(  a  \)对模 \(  m  \)的指数.    

    若 \(  a  \)对模 \(  m  \)的指数是 \(  \varphi \left( m \right)   \),则\(  a  \)叫做模 \(  m  \)的一个原根.     
\end{definition}

\begin{theorem}
    若 \(  a  \)对模 \(  m  \)的指数为 \(   \delta   \),则 \(  1=  a^0,a^1,\cdots,a^{\delta -1}  \)对模 \(  m  \)两两不同余.     
\end{theorem}

\begin{theorem}
    若 \(  a  \)对模 \(  m  \)的指数是 \(   \delta   \),则 \(  a^{ \gamma } =  a^{ \gamma ^{\prime} }\left( \operatorname{mod}\,m \right)   \)   当且仅当 \(   \gamma =  \gamma ^{\prime} \left( \operatorname{mod}\,\delta  \right)   \).
    
    特别地, \(  a^{ \gamma }\equiv 1\left( \operatorname{mod}\, \delta  \right)   \) 当且仅当 \( \delta \mid   \gamma   \) .
\end{theorem}

\begin{theorem}
   \begin{enumerate}
    \item  若 \(  x  \)对模 \(  m  \)的指数是 \(  ab  \), \(  a>0,b>0  \),则 \(  x^{a}  \)对模 \(  m  \)的指数是 \(  b  \).       
    \item 若 \(  x  \)对模 \(  m  \)的指数是 \(  a  \),\(  y  \)对模 \(  m  \)的指数是\(  b  \),且 \(  \left( a,b \right)= 1   \)      ,则 \(  xy  \)对模 \(  m  \)的指数是 \(  ab  \).   
   \end{enumerate}
   
\end{theorem}

\section{原根存在条件}

\begin{theorem}
     \[
     \text{模}m\text{的原根存在} \iff  m= 2,4,p^{\alpha },2p^{\alpha },\quad p\text{是奇素数}
     \] 
\end{theorem}

\section{指标和 \(  n  \)次剩余  }

考察同余式 \[
x^{n}\equiv a\left( \operatorname{mod}\,m \right),\quad \left( a,m \right)= 1  
\]解的存在条件,解的个数,模 \(  m  \)的原根的个数. 

若无特别声明,以下皆设 \(  m   \)是 \(  p^{ \alpha }  \)或 \(  2p^{ \alpha }  \),\(  c =  \varphi \left( m \right)   \), \(  g  \)是 模 \(  m  \)的一个原根.


\begin{theorem}
    若 \(   \gamma   \)过模 \(  c  \)的最小非负完全剩余系, 则 \(  g^{ \gamma }  \)过模 \(  m  \)的一个既约剩余系.   
\end{theorem}


\begin{definition}
    设 \(  a  \)是整数,若对于模 \(  m  \)的一个原根\(  g  \) ,存在整数 \(   \gamma   \),使得 \[
    a \equiv g^{ \gamma }\left( \operatorname{mod}\,m \right) ,\quad \gamma \ge 0
    \]   则 称 \(   \gamma   \)为以 \(  g  \)为底的 \(  a  \)对模 \(  m  \)的一个指标.    
\end{definition}

\begin{remark}
    可以将指标 \(   \gamma   \)看成是 \(  a  \)以 \(  g  \)为底的对数,只不过这里 \(   \gamma   \)只在模 \(  m  \)的意义下唯一 .   
\end{remark}

\begin{theorem}
    设 \(  a  \)是整数使得\(  \left( a,m \right)= 1   \), \(  g  \)是模 \(  m  \)的一个原根.则存在 \(   \gamma ^{\prime}   \)满足 \(  0\le  \gamma ^{\prime} <c  \),是一个 \(  a  \)的以 \(  g  \)为底的模 \(  m  \)的指标.
    
    此外,整数 \(   \gamma   \)是  \(  g  \)为底的 \(  a  \)的模 \(  m  \)的指标,当且仅当它满足 \[
     \gamma \equiv  \gamma ^{\prime} \left( \operatorname{mod}\,c \right),\quad  \gamma \ge 0 
    \]此 \(   \gamma ^{\prime}   \)记作 \(  \operatorname{ind}\,_{g}a  \)或 \(  \left( \operatorname{ind}\,a \right)   \)   
\end{theorem}

\begin{theorem}
    设 \(  g  \)是模 \(  m  \)的一个原根, \(   \gamma   \)是一个非负整数.则以 \(  g  \)为底,对模 \(  m  \)有同一指标的 \(   \gamma   \)的一切指标构成模 \(  m  \)的一个与模互素的剩余类.      
\end{theorem}


\begin{theorem}
    设 \(   a_1,\cdots,a_n   \)是与 \(  m  \)互素的 \(  n  \)个整数,则 \[
    \operatorname{ind}\,\left( a_1a_2\cdots a_{n} \right) \equiv  \operatorname{ind}\,a_1+ \operatorname{ind}\,a_2+ \cdots + \operatorname{ind}\,a_{n}\left( \operatorname{mod}\,c \right)  
    \]   
\end{theorem}
\end{document}
