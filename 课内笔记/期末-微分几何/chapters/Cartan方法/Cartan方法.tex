\documentclass[../../main.tex]{subfiles}

\begin{document}

\chapter{ Cartan方法 }

若无特别指出,本章采用以下约定:

\begin{enumerate}
    \item \(  \left( M,g \right)   \)是一个\(  n  \)维Riemann流形.   
    \item \(   \nabla   \)是 \(  TM  \)上的Levi-Civita联络.
    \item \(  U  \)是 \(  M  \)上的一个开子集, \(  \left( E_{i} \right)   \)是\(  U  \)上的一组局部标架, \(  \left(  \varepsilon ^{i} \right)   \)是对偶的余标架.       
\end{enumerate}
\section{基本概念}

\begin{definition}{1-形式的内积}
    设 \(  \alpha = \alpha _{i}\,\mathrm{d} x^{i}  \)和 \(  \beta = \beta _{j}\,\mathrm{d} x^{j}  \)  是两个1-形式,定义它们的内积为分别提升指标后向量场的内积,即 \[
    \left<\alpha ,\beta  \right>_{g}= \left<g^{ik}\alpha _{k}\frac{\partial }{\partial x^{i}},g^{jl}\beta _{l}\frac{\partial }{\partial x^{j}} \right>=  g^{kl}\alpha _{k}\beta _{l}
    \]
\end{definition}

\begin{definition}{联络1-形式}
    \(  U  \)上存在唯一的光滑 \(  1  \)-形式的\(  n\times n  \)矩阵 \(  \left(  \omega _{i}^{j} \right)   \),使得 \[
     \nabla _{X}E_{i}=  \omega _{i}^{j}\left( X \right)E_{j},\quad \forall  X \in \mathfrak{X}\left( U \right)  
    \]或者写作 \[
     \nabla E_{i}=  \omega _{i}^{j}\otimes E_{j}
    \]称为是这组标架的\textbf{联络1-形式} .   
\end{definition}
\begin{proof}
     若存在这样的 \(  1  \)-形式 \(   \omega   \),则 \[
   \Gamma _{ij}^{k}E_{k}=    \nabla _{E_{i}}E_{j}=  \omega _{j}^{k}\left( E_{i} \right)E_{k} 
    \]  得到 \(   \omega _{j}^{k}\left( E_{i} \right)=  \Gamma _{ij}^{k},\forall i,j,k   \) .于是我们定义 \[
     \omega _{i}^{j}\left( X \right)=X^{k} \Gamma _{ki}^{j},\quad \forall X = X^{k}E_{k} \in \mathfrak{X}\left( U \right) 
    \]则由 \(   \Gamma _{ki}^{j}  \)的光滑性, \(   \omega _{i}^{j}  \)是光滑的余标架.对于任意的 \(  X =  X^{k}E_{k} \in \mathfrak{X}\left( U \right)   \), \[
     \nabla _{X}E_{i}= X^{k} \nabla _{E_{k}}E_{i}= X^{k} \Gamma _{ki}^{l}E_{l}=  \omega _{i}^{l}\left( X \right)E_{l}=  \omega _{i}^{j}\left( X \right) E_{j} 
    \]   

    \hfill $\square$
\end{proof}


\begin{definition}{曲率2-形式}
    按以下方式定义一个2-形式的矩阵 \(  \left(  \Omega _{i}^{j} \right)   \)\[
     \Omega _{i}^{j}= \frac{1}{2}R_{kli}^{j} \varepsilon ^{k}\wedge  \varepsilon ^{l}
    \] 称为是\textbf{曲率2-形式}
\end{definition}

\section{结构方程}

\begin{theorem}{Cartan结构方程}
    以下两个Cartan结构方程成立
    \begin{enumerate}
        \item  \[
        \,\mathrm{d}  \varepsilon ^{j}=  \varepsilon ^{i}\wedge  \omega _{i}^{j}
        \]
        \item  \[
   \Omega _{i}^{j}= \,\mathrm{d} { \omega _{i}}^{j} - \omega _{i}^{k}\wedge  \omega _{k}^{j}
  \]  
    \end{enumerate}
    
\end{theorem}
\begin{proof}
    \begin{enumerate}
        \item 一方面 \[
    \begin{aligned}
    \,\mathrm{d}  \varepsilon ^{j}\left( E_{k},E_{l} \right)&=   E_{k}\left(  \varepsilon ^{j}\left( E_{l} \right)  \right)- E_{l}\left(  \varepsilon ^{j}\left( E_{k} \right)  \right) - \varepsilon ^{j}\left( \left[ E_{k},E_{l} \right]  \right) = - \varepsilon ^{j}\left( \left[ E_{k},E_{l} \right]  \right) 
    \end{aligned}
    \] 另一方面 \[
    \begin{aligned}
 \left(     \varepsilon ^{i}\wedge  \omega _{i}^{j}\right)\left( E_{k},E_{l} \right)&=    \varepsilon ^{i}\left( E_{k} \right) \omega _{i}^{j}\left( E_{l} \right)-  \varepsilon  ^{i}\left( E_{l} \right) \omega _{i}^{j}\left( E_{k} \right)\\ 
  &=  \omega _{k}^{j}\left( E_{l} \right)- \omega _{l}^{j}  \left( E_{k} \right)     \\ 
   &=  \varepsilon ^{j}\left(  \nabla _{E_{l}} E_{k}\right)- \varepsilon ^{j}\left(  \nabla _{E_{k}}E_{l} \right)\\ 
    &=  \varepsilon ^{j}\left<  \nabla _{E_{l}}E_{k}- \nabla _{E_{k}}E_{l} \right>  \\ 
     &= - \varepsilon ^{j}\left( \left[ E_{k},E_{l} \right]  \right) 
    \end{aligned}
    \]
    \item  \[
   \nabla _{X}E_{i}=  \omega _{i}^{j}\left( X \right)E_{j} 
  \]  \[
   \Gamma _{ki}^{j}E_{j}=  \nabla _{E_{k}}E_{i}=  \omega _{i}^{j}\left( E_{k} \right)E_{j} 
  \]故 \[
   \omega _{i}^{j}\left( E_{k} \right)=  \Gamma _{ki}^{j} 
  \] \[
 \begin{aligned}
 \,\mathrm{d}  \omega _{i}^{j}\left( E_{k},E_{l} \right)&=  E_{k}\left(  \omega _{i}^{j}\left( E_{l} \right)  \right)-E_{l}\left(  \omega _{i}^{j}\left( E_{k} \right)  \right)- \omega _{i}^{j}\left( \left[ E_{k},E_{l} \right]  \right)   
 \end{aligned}
  \]\[
   \omega _{i}^{k}\wedge  \omega _{k}^{j}\left( E_{k},E_{l} \right)= \omega _{i}^{m}\left( E_{k} \right) \omega _{m}^{j}\left( E_{l} \right)   - \omega _{i}^{m}\left( E_{l} \right)  \omega _{m}^{j}\left( E_{k} \right) 
  \]\[
  \begin{aligned}
  R_{kli}&=  \nabla _{E_{k}}  \nabla _{E_{l}}E_{i}- \nabla _{E_{l}} \nabla _{E_{k}}E_{i}- \nabla _{\left[ E_{k},E_{l} \right] }E_{i}\\ 
   &=  \nabla _{E_{k}}\left(  \omega _{i}^{j}\left( E_{l} \right)E_{j}  \right) - \nabla _{E_{l}}\left(  \omega _{i}^{j}\left( E_{k} \right)E_{j}  \right)-  \omega _{i}^{j}\left( \left[  E_{k},E_{l}\right]  \right)E_{j}  \\ 
    &=  \omega _{i}^{j}\left( E_{l} \right) \nabla _{E_{k}}E_{j} + E_{k}\left(  \omega _{i}^{j}\left( E_{l} \right)  \right)E_{j} -  \omega _{i}^{j}\left( E_{k} \right) \nabla _{E_{l}} E_{j}- E_{l}\left(  \omega _{i}^{j}\left( E_{k} \right)  \right)E_{j}\\ 
     &- \omega _{i}^{j}\left( \left[ E_{k},E_{l} \right]  \right)E_{j}  \\ 
      &=  \omega _{i}^{j}\left( E_{l} \right) \omega _{j}^{m}\left( E_{k} \right)E_{m} + E_{k}  \left(  \omega _{i}^{j}\left( E_{l} \right)  \right)E_{j}-  \omega _{i}^{j}\left( E_{k} \right) \omega _{j}^{m}\left( E_{l} \right)E_{m}-E_{l}\left(  \omega _{i}^{j}\left( E_{k} \right)  \right)E_{j}\\ 
       &- \omega _{i}^{j}\left( \left[ E_{k},E_{l} \right]  \right)E_{j}     \\ 
        &=   \omega _{i}^{m}\left( E_{l} \right) \omega _{m}^{j}\left( E_{k} \right)E_{j}+ E_{k}\left(  \omega _{i}^{j}\left( E_{l} \right)  \right)E_{j}- \omega _{i}^{m}\left( E_{k} \right) \omega _{m}^{j}\left( E_{l} \right)E_{j} -E_{l}\left(  \omega _{i}^{j}\left( E_{k} \right)  \right)E_{j}\\ 
         &- \omega _{i}^{j}\left( \left[  E_{k},E_{l} \right]  \right)E_{j}       
  \end{aligned}
  \]于是 \[
  \begin{aligned}
  R_{kli}^{j}&=  \omega _{i}^{m}\left( E_{l} \right) \omega _{m}^{j}\left( E_{k} \right)+ E_{k}\left(  \omega _{i}^{j}\left( E_{l} \right)  \right)- \omega _{i}^{m}\left( E_{k} \right) \omega _{m}^{j}\left( E_{l} \right)-E_{l}\left(  \omega _{i}^{j}\left( E_{k} \right)  \right)\\ 
   &- \omega _{i}^{j}\left( \left[ E_{k},E_{l} \right]  \right)
  \end{aligned}
  \] 
\[
 \Omega _{i}^{j}= \frac{1}{2}R_{kli}^{j}\left(  \varepsilon ^{k}\otimes  \varepsilon ^{l}- \varepsilon ^{l}\otimes  \varepsilon ^{k} \right) = \sum _{k<l}R_{kli}^{j} \varepsilon ^{k}\otimes  \varepsilon ^{l}
\]于是 \[
 \Omega _{i}^{j}\left( E_{k},E_{l} \right)= R_{kli}^{j} = \,\mathrm{d}  \omega _{i}^{j}\left( E_{k},E_{l} \right)-  \omega _{i}^{k}\wedge  \omega _{k}^{j}\left( E_{k},E_{l} \right)  
\]

    \end{enumerate}
    

    \hfill $\square$
\end{proof}

\section{规正标价}

\begin{proposition}
    若 \(  \left(  \varepsilon ^{i} \right)   \)是规正的余标架.则黎曼度量 \(  g  \)在局部上表示为 \[
    g = \left(  \varepsilon ^{1} \right)^{2}+ \cdots + \left(  \varepsilon ^{n} \right)^{2}  
    \]  
\end{proposition}

\begin{proposition}
    若 \(  \left(  \varepsilon ^{i} \right)   \)是正交的余标架,则 \[
     \omega _{i}^{j}= - \omega _{j}^{i}
    \]
\end{proposition}
\begin{proof}
    \[
  0=   \nabla _{X}\left< \varepsilon ^{i}, \varepsilon ^{j} \right>= \left< \nabla_{X}  \varepsilon ^{i}, \varepsilon ^{j} \right>+ \left< \varepsilon ^{i}, \nabla _{X} \varepsilon ^{j} \right>
    \] 其中由正交性, \[
    \left< \nabla _{X} \varepsilon ^{i}, \varepsilon ^{j} \right>= \left< \nabla _{X}E_{i},E_{j} \right>= \omega _{i}^{j}\left( X \right),\quad \left< \varepsilon ^{i}, \nabla _{X} \varepsilon ^{j} \right>= \left<E_{i}, \nabla _{X}E_{j} \right>=  \omega _{j}^{i}\left( X \right)   
    \]于是 \[
     \omega _{i}^{j}= - \omega _{j}^{i}
    \]

    \hfill $\square$
\end{proof}

\section{超曲面}

约定 \(  N  \)是 \(  M  \)中余1-维的超曲面 , \(   E_1,\cdots,E_n   \)是 \(  M  \)的一个局部规正标架,其中 \(  E_1,\cdots ,E_{n-1}  \)是 \(  N  \)的切向量, \(  E_{n}  \)是\(  N  \)的单位法向量       .相应地, \(  \varepsilon ^{1},\cdots , \varepsilon ^{n-1}  \)是 \(  N  \)的切规正余标架, \(   \varepsilon ^{n}  \)是法向余标架.       

\begin{theorem}
    设 \(  h  \)是 \(  N  \)的标量第二基本形式,则 \[
    h_{ij}=  \omega _{j}^{n}\left( E_{i} \right),\quad i,j\in \left\{ 1,\cdots ,n-1 \right\} 
    \]  
\end{theorem}
\begin{proof}
    \[
    h_{ij}= h\left( E_{i},E_{j} \right)= \left< \nabla _{E_{i}}E_{j},E_{n} \right>= \left< \omega _{j}^{k}\left( E_{i} \right)E_{k},E_{n}  \right> =  \omega _{j}^{n}\left( E_{i} \right) 
    \]

    \hfill $\square$
\end{proof}

\begin{theorem}{Weingarten}
    设 \(  S  \)是Weingarten变换,则 \[
    S\left( X \right)= \sum _{i= 1}^{n-1} \omega _{i}^{n}\left( X \right)E_{i}  
    \]\[
    S_{ij}=  \omega _{j}^{n}\left( E_{i} \right)= h_{ij} 
    \]
\end{theorem}


\begin{theorem}{Gauss方程}
    设 \(   \Omega ^{N}  \)是 \(  N  \)上诱导度量的曲率2-形式.则 \[
     \Omega _{i}^{j}=  \Omega _{i}^{j,N}+  \omega _{i}^{n}\wedge  \omega _{j}^{n}
    \]特别地,若 \(  M  \)是平坦的(比如欧式空间),则 \[
    0=  \Omega _{i}^{j,N}+  \omega _{i}^{n}\wedge  \omega _{j}^{n}
    \] 
\end{theorem}
\begin{proof}
    考虑 \(  N  \)上的第二结构方程 \[
     \Omega _{i}^{j,N}= \,\mathrm{d}  \omega _{i}^{j}- \sum _{k= 1}^{n-1}\omega _{i}^{k}\wedge  \omega _{k}^{j},\quad i,j\in \left\{ 1,2,\cdots ,n-1 \right\}
    \] 与\(  M  \)上的第二结构方程 \[
     \Omega _{i}^{j}= \,\mathrm{d}  \omega _{i}^{j}-\sum _{k= 1}^{n} \omega _{i}^{k}\wedge  \omega _{k}^{j}
    \] 相减并利用反对称性

    \hfill $\square$
\end{proof}

\begin{lemma}{Gauss曲率}
   设 \(  M  \)是3维欧式空间,\(  K  \)是\(  N  \)在 \(  M  \)中的Gauss曲率,则     \[
     \Omega _{2}^{1,N}= K \varepsilon ^{1}\wedge  \varepsilon ^{2}
    \]
\end{lemma}
\begin{proof}
    根据定义 \[
     \Omega _{2}^{1,N}= \frac{1}{2}R_{122}^{1} \varepsilon ^{1}\wedge  \varepsilon ^{2}+ \frac{1}{2}R_{212}^{1} \varepsilon ^{2}\wedge  \varepsilon ^{1}
    \]由曲率张量的对称性和标架的正交性, \[
    R_{122}^{1}= -R_{212}^{1}= R_{1221}= K
    \]于是 \[
     \Omega _{2}^{1,N}= \frac{1}{2}K \varepsilon ^{1}\wedge  \varepsilon ^{2}-\frac{1}{2}K \varepsilon ^{2}\wedge  \varepsilon ^{1}= K \varepsilon ^{1}\wedge  \varepsilon ^{2}
    \]

    \hfill $\square$
\end{proof}

\begin{corollary}
    对于二维曲面 \(  N  \), 正交标价下的第二结构方程简化为  \[
     \Omega _{i}^{j}= \,\mathrm{d}  \omega _{i}^{j}
    \]
\end{corollary}
\begin{proof}
    结构方程中 \(   \omega _{i}^{k}\wedge  \omega _{k}^{j}  \)中的每一项都含对角元,而正交标价下联络1-形式矩阵的对角元为零. 

    \hfill $\square$
\end{proof}
\begin{corollary}
    设 \(  M  \)是3维欧式空间,\(  K  \)是\(  N  \)在 \(  M  \)中的Gauss曲率,则 \[
    K \varepsilon ^{1}\wedge  \varepsilon ^{2}= \,\mathrm{d}  \omega _{2}^{1}
    \]
\end{corollary}

\section{计算}

\subsection{借助氛围欧式空间的计算}
主要是利用适配标架,在欧式空间上计算,再带入到子流形上获得子流形上几何量.
\begin{method}{联络形式的计算方法}
    \begin{enumerate}
        \item 计算坐标/参数向量场.
        \item 对坐标/参数向量场进行正交化,得到规正的切丛的标架.
        \item 计算全协变导数 \[
         \nabla E_{i}
        \]结果是一个 \(  \left( 1,1 \right)   \)-张量,通常用欧式空间上的标准向量场 \(   \partial _{i}  \),以及坐标余向量场 \(  \,\mathrm{d} r_{i}  \)表示.   计算的过程中,使用Lebniz律,以及事实 :\[
     \nabla f= \,\mathrm{d} f,\quad f\in C^{\infty}\left( M \right) 
    \]
    \item 根据定义 \[
     \nabla E_{i}=  \omega _{i}^{j}\otimes E_{j}
    \]通过将 \(   \nabla E_{i}  \)与 \(  E_{l}  \)做度量配对\(  \left< \nabla E_{i},E_{l} \right>_{g}  \) ,得到 \(   \omega _{i}^{l}  \).   
    \end{enumerate}
    
\end{method}

\begin{method}{Gauss曲率的计算方法}
    \begin{enumerate}
        \item 根据上面的方法,计算 \[
     \omega _{1}^{2}= \left< \nabla E_{1},E_2 \right>_{g}
    \]或者 \[
      \omega _{2}^{1}= \left< \nabla E_2,E_1 \right>_{g}
    \]
    \item 计算外微分 \[
    \,\mathrm{d}  \omega _{2}^{1}= -\,\mathrm{d}  \omega _{1}^{2}
    \]
    \item 利用简化的Gauss方程 \[
    K \varepsilon ^{1}\wedge  \varepsilon ^{2}= \,\mathrm{d}  \omega _{2}^{1}
    \]
    \item 两边按相同的基表示,对比得到 \(  K  \).
    \end{enumerate}
    
\end{method}
\begin{example}[ 球面]
    计算半径为 \(  R  \)的球面的Gauss曲率    
\end{example}

\begin{solution}
    考虑参数化 \[
    r\left(  \theta , \varphi  \right)= \left( R\sin  \varphi \cos  \theta ,R\sin  \varphi \sin  \theta ,R\cos  \varphi  \right)  
    \] \[
    r_{ \theta }= R\left( -\sin  \varphi \sin  \theta ,\sin  \varphi \cos  \theta ,0 \right),\quad r_{ \varphi }= R\left( \cos  \varphi \cos  \theta ,\cos  \varphi \sin  \theta ,-\sin  \varphi  \right)  
    \]则 \[
    r_{ \theta }\cdot r_{ \varphi }= 0
    \]令 \[
    E_1= \frac{r_{ \theta } }{\left| r_{ \theta } \right|  }= \left( -\sin  \theta ,\cos  \theta ,0 \right)  ,\quad E_2= \frac{r_{ \varphi } }{\left| r_{ \varphi } \right|  }= \left( \cos  \varphi \cos  \theta ,\cos  \varphi \sin  \theta ,-\sin  \varphi  \right)  
    \]令 \[
    E_3= E_1\times E_2= \begin{pmatrix} 
        i&j&k\\ 
         -\sin  \theta &\cos  \theta &0\\ 
          \cos  \varphi \cos  \theta &\cos  \varphi \sin  \theta &-\sin  \varphi   
    \end{pmatrix}= \left( -\sin  \varphi \cos  \theta  ,-\sin  \varphi \sin  \theta  , -\cos  \varphi \right)  
    \]记球面为 \(  \mathbb{S}  \),则 \[
    E_1,E_2\in T \mathbb{S},\quad E_3\in N \mathbb{S}
    \]   \[
     \varepsilon ^{1}= \left| r_{ \theta } \right|\,\mathrm{d}  \theta = R\sin  \varphi \,\mathrm{d}  \theta ,\quad  \varepsilon ^{2}= \left| r_{ \varphi } \right|\,\mathrm{d}  \varphi =R \,\mathrm{d}  \varphi   
    \]\[
  \begin{aligned}
     \nabla E_1=    \nabla \left( -\sin  \theta  \partial _{1}+ \cos  \theta  \partial _{2} \right)&=  -\cos  \theta  \,\mathrm{d}  \theta \otimes  \partial _{1}-\sin  \theta \,\mathrm{d}  \theta \otimes  \partial _{2}
  \end{aligned}
    \]其中, \(   \partial _{1}, \partial _{2}, \partial _{3}  \)表示 \(  \mathbb{R} ^{3}  \)上的标准坐标向量场.  另一方面 \[
     \nabla E_1=  \omega _{1}^{1}\otimes E_1+  \omega _{1}^{2}\otimes E_2+  \omega _{1}^{3}\otimes E_3=  \omega _{1}^{2}\otimes E_2+  \omega _{1}^{3}\otimes E_3
    \]  \[
     \omega _{1}^{2}= -\cos  \theta \,\mathrm{d}  \theta \left< \partial _{1},E_2 \right>-\sin  \theta \,\mathrm{d}  \theta \left< \partial _{2} ,E_2\right>= -\cos  \varphi \,\mathrm{d}  \theta 
    \]于是 \[
     \omega _{2}^{1}= \cos  \varphi \,\mathrm{d}  \theta 
    \]
    

    进而 \[
    \,\mathrm{d}  \omega _{2}^{1}=\sin  \varphi \,\mathrm{d}  \theta \wedge \,\mathrm{d}  \varphi = \frac{1 }{R\sin  \varphi  }\sin  \varphi  \frac{1 }{R } \varepsilon ^{1}\wedge  \varepsilon ^{2}= \frac{1 }{R^{2} }  \varepsilon ^{1}\wedge  \varepsilon ^{2} 
    \]由Gauss方程 \[
    K \varepsilon ^{1}\wedge  \varepsilon ^{2}=  \,\mathrm{d}  \omega _{2}^{1}
    \]得到 \(  K= \frac{1 }{R^{2} }   \) 
\end{solution}

\hspace*{\fill} 
\begin{example}
    设曲面 $S: \mathbf{r} = \mathbf{r}(u,v)$ 上没有抛物点, $\mathbf{n}$ 是 $S$ 的法向量; 曲面 $\tilde{S}: \tilde{\mathbf{r}}(u,v) = \mathbf{r}(u,v) + \lambda \mathbf{n}(u,v)$ (常数 $\lambda$ 充分小) 称为 $S$ 的平行曲面.
\begin{enumerate}
    \item 证明曲面 $S$ 和 $\tilde{S}$ 在对应点的切平面平行;
    \item 可以选取 $\tilde{S}$ 的单位法向 $\tilde{\mathbf{n}}$, 使得 $\tilde{S}$ 的 Gauss 曲率和平均曲率分别为
    $$ \tilde{K} = \frac{K}{1 - 2\lambda H + \lambda^2 K}, \quad \tilde{H} = \frac{H - \lambda K}{1 - 2\lambda H + \lambda^2 K}. $$
\end{enumerate}
\end{example}

\begin{solution}
    \begin{enumerate}
        \item \[
        \tilde{\mathbf{r}}_{u}= \mathbf{r}_{u}+  \lambda \mathbf{n}_{u},\quad \tilde{\mathbf{r}}_{v}= \mathbf{r}_{v}+  \lambda \mathbf{n}_{v}
        \] 由Weingarten方程,\[
         \mathbf{n}_{u}=  \nabla ^{g} _{\mathbf{r}_{u}}\mathbf{n}= -s\left( \mathbf{r}_{u} \right)\in \operatorname{span}\,\left( \mathbf{r}_{u},\mathbf{r}_{v} \right) 
        \]类似地 \[
        \mathbf{n}_{v}\in \operatorname{span}\,\left( \mathbf{r}_{u},\mathbf{r}_{v} \right) 
        \]从而 \[
        T_{\tilde{\mathbf{r}}\left( u,v \right) }\tilde{S}= \operatorname{span}\,\left( \mathbf{\tilde{r}}_{u}, \mathbf{\tilde{r}}_{v} \right)= \operatorname{span}\,\left( \mathbf{r}_{u},\mathbf{r}_{v} \right)  = T_{\mathbf{r}\left( u,v \right) }S
        \]这表明\(  S  \)在 \(  \mathbf{r}\left( u,v \right)   \)处的切平面与 \(  \tilde{S}  \)在 \(  \tilde{\mathbf{r}}\left( u,v \right)   \)处的切平面平行.
        \item  \[
        g = \left<r_{u},r_{u} \right>\,\mathrm{d} u\otimes \,\mathrm{d} u+ 2\left<r_{u},r_{v} \right>\,\mathrm{d} u\otimes \,\mathrm{d} v+ \left<r_{v},r_{v} \right> \,\mathrm{d} v\otimes \,\mathrm{d} v
        \] \[
        \tilde{g}= \left<\tilde{r}_{u},\tilde{r}_{v} \right>\,\mathrm{d} u\otimes \,\mathrm{d} v+ 2\left<\tilde{r}_{u},\tilde{r}_{v} \right>\,\mathrm{d} u\otimes \,\mathrm{d} v+ \left<\tilde{r}_{v},\tilde{r}_{v} \right>\,\mathrm{d} v\otimes \,\mathrm{d} v
        \]注意到 \[
        \tilde{r}_{ \Lambda  }= \tilde{r}_{ \Lambda }+  \lambda n_{ \Lambda }= r_{ \Lambda }- \lambda S\left( r_{ \Lambda } \right)= \left( \mathrm{Id}- \lambda S \right)r_{ \Lambda },\quad  \Lambda = u,v  
        \]于是 \[
        \left<\tilde{r}_{i},\tilde{r}_{j} \right>= \det \left( \operatorname{Id}- \lambda S \right) ^{2}\left<\tilde{r}_{j},\tilde{r}_{j} \right>
        \] 进而 \[
        \tilde{g}= \det \left( Id-  \lambda S \right)^{2}\det g 
        \]\[
         \omega _{2}^{1}= \left< \nabla e_{2},e_1 \right>= \left< \nabla ^{g}e_2-h\left( \cdot ,e_2 \right)e_3,e_1  \right>= \left< \nabla ^{g}e_2,e_1 \right>
        \]类似地 \[
         \tilde{\omega} _{2}^{1}= \left< \nabla ^{g}e_2,e_1 \right>
        \]因此 \[
         \omega _{2}^{1}=  \tilde{\omega}_{2}^{1}
        \]进而 \[
        \,\mathrm{d}  \omega _{2}^{1}= \,\mathrm{d}  \tilde{\omega} _{2}^{1}
        \]于是 \[
        K  \varepsilon ^{1}\wedge  \varepsilon ^{2}= \tilde{K} \tilde{\varepsilon} ^{1}\wedge  \tilde{\varepsilon} ^{2}
        \]其中 \(   \tilde{\varepsilon} ^{i}  \)是 \(  e_{i}  \)关于 \(  \tilde{g}  \)的对偶余标架, \(   \varepsilon ^{i}  \)是 \(  e_{i}  \)关于 \(  g  \)的对偶余标架.那么 \[
         \tilde{\varepsilon} ^{1}\wedge  \tilde{\varepsilon} ^{2}= \sqrt{\tilde{g} } \,\mathrm{d} u\,\mathrm{d} v,\quad   \varepsilon ^{1}\wedge  \varepsilon ^{2}= \sqrt{g}\,\mathrm{d} u\,\mathrm{d} v
        \]        由于 \[
        \det \tilde{g} = \det \left( I-  \lambda S \right)^{2}\det g 
        \]故 \[
         \tilde{\varepsilon} ^{1}\wedge  \tilde{\varepsilon} ^{2}= \det \left( I- \lambda S \right) \varepsilon ^{1}\wedge  \varepsilon ^{1} 
        \]进而 \[
        \tilde{K}= \frac{K }{\det \left( I- \lambda S \right)  } 
        \]在 \(  e_1,e_2  \)下, \[
        S= \begin{pmatrix} 
             \kappa _1 &0\\ 
              0& \kappa _2  
        \end{pmatrix} 
        \] 于是 \[
        \det \left( I- \lambda S \right)= \det \begin{pmatrix} 
            1-  \lambda \kappa _1 &0\\ 
             0&1-  \lambda \kappa _2  
        \end{pmatrix}= \left( 1-  \lambda \kappa _1 -  \lambda \kappa _2 -  \lambda ^{2}\kappa _1  \kappa _2  \right)=    1-2 \lambda H- \lambda ^{2}K
        \]最终 \[
        \tilde{K}= \frac{K }{1-2 \lambda H+  \lambda ^{2}K } 
        \]

    \end{enumerate}
    
\end{solution}

\hspace*{\fill} 

\begin{problem}
设曲面 $S$ 由方程 $x^2 + y^2 - f(z) = 0$ 给定, $f$ 满足 $f(0) = 0$, $f'(0) \neq 0$, 证明: $S$ 在点 $(0,0,0)$ 的法曲率为常数.
\end{problem}
\begin{proof}
   令 \[
   F\left( x,y,z \right)= x^{2}+ y^{2}-f\left( z \right)  
   \],则 \[
    \operatorname{grad}\, F= \left( 2x,2y,-f^{\prime} \left( z \right)  \right) 
   \] 令 \[
   e_3= \frac{ \operatorname{grad}\,F }{\left| \operatorname{grad}\, F \right|  }= \frac{\left( 2x,2y,-f^{\prime} \left( z \right)  \right)  }{\sqrt{4x^{2}+ 4y^{2}+ \left( f^{\prime} \left( z \right)  \right)^{2} } }  
   \]设 \(   \nabla   \)是 \(  S  \)上的协变导数, 令 \(  N\left( x,y,z \right)= \sqrt{4x^{2}+ 4y^{2}+ \left( f^{\prime} \left( z \right)^{2}  \right) }   \) 
   则 \[
    \begin{aligned}
    \nabla e_3&=  \nabla \left( \frac{1}{N}\left( 2x \partial _{1}+ 2y \partial _{2}-f^{\prime} \left( z \right) \partial _{3}  \right)  \right) \\ 
     &= d\left( \frac{1 }{N }  \right)\otimes \left( 2x \partial _{1}+ 2y \partial _{2}-f^{\prime} \left( z \right) \partial _{3}  \right)+ \frac{1 }{N } \left( 2\,\mathrm{d} x\otimes  \partial _{1}+ 2\,\mathrm{d} y\otimes  \partial _{2}-f^{\prime \prime} \left( z \right)\,\mathrm{d} z\otimes  \partial _{3}  \right)     
    \end{aligned}
   \]其中 \[
   \,\mathrm{d} \left( \frac{1 }{N }  \right)= -\frac{\,\mathrm{d} N }{N^{2} }  
   \] \[
   \,\mathrm{d} N= \frac{1 }{2N }\left( 8x\,\mathrm{d} x+ 8y\,\mathrm{d} y+ 2f^{\prime} \left( z \right)f^{\prime \prime} \left( z \right)\,\mathrm{d} z   \right)  
   \]从而 \[
  \begin{aligned}
    \nabla e_3&= -\frac{1 }{2N^{3} }\left( 8x\,\mathrm{d} x+ 8y\,\mathrm{d} y+ 2f^{\prime} \left( z \right)f^{\prime \prime} \left( z \right)\,\mathrm{d} z   \right)\otimes \left( 2x \partial _{1}+ 2y \partial _{2}-f^{\prime} \left( z \right) \partial _{3}  \right)  \\ 
     & + \frac{1 }{N }\left( 2\,\mathrm{d} x\otimes  \partial _{1}+ 2\,\mathrm{d} y\otimes  \partial _{2}-f^{\prime \prime} \left( z \right)\,\mathrm{d} z\otimes  \partial _{3}  \right)  
  \end{aligned}
   \]选取 \(  e_1,e_2  \),使得 \(  e_1|_{0}= \left( 1,0,0 \right)=  \partial _{1}, e_2|_{0}= \left( 0,1,0 \right)  =  \partial _{2}  \), \(   \varepsilon ^{1}, \varepsilon ^{2}  \)分别是 \(  e_1,e_2  \)的对偶余向量场.那么 \[
   \operatorname{II} =  \omega _{1}^{3} \otimes \varepsilon ^{1}+  \omega _{2}^{3} \otimes \varepsilon ^{2}
   \]   在原点处,\(  N\left( 0,0,0 \right)= f^{\prime} \left( 0 \right)    \)  进而 \[
    \nabla e_3= \frac{1 }{f^{\prime} \left( 0 \right)   }f^{\prime \prime} \left( 0 \right)\,\mathrm{d} z\otimes    \partial _{3}+ \frac{1 }{f^{\prime} \left( 0 \right)  } \left( 2\,\mathrm{d} x\otimes  \partial _{1}+ 2\,\mathrm{d} y\otimes  \partial _{2}-f^{\prime \prime} \left( 0 \right)\,\mathrm{d} z\otimes  \partial _{3}  \right) 
   \] 在原点处成立.紧接着就有 \[
    \omega _{3}^{1}= \left< \nabla e_3,e_1 \right>= \frac{1 }{f^{\prime} \left( 0 \right)  }2 \,\mathrm{d} x,\quad  \omega _{3}^{2}= \left< \nabla e_3,e_2 \right>= \frac{1 }{f^{\prime} \left( 0 \right)  } 2\,\mathrm{d} y  
   \]在原点处成立,其中尖括号表示关于两个向量场的缩并.又\[
   \,\mathrm{d} x=  \varepsilon ^{1},\quad \,\mathrm{d} y=  \varepsilon ^{2}
   \]在原点处成立.于是 \[
   \operatorname{II} _{0}= \left( \frac{2 }{f^{\prime} \left( 0 \right)  } \left(   \varepsilon ^{1}\otimes   \varepsilon ^{1} \right)+ \frac{2 }{f^{\prime} \left( 0 \right)  }\varepsilon ^{2}\otimes   \varepsilon ^{2}    \right)_{0}=  \frac{2 }{f^{\prime} \left( 0 \right)  } \operatorname{Id}_{0}  
   \]第二基本形式在原点处为数量矩阵,从而法曲率在原点处为常数.
\end{proof}

\subsection{内蕴解法}

\begin{method}
    \begin{enumerate}
        \item 写出一组内蕴的正交余标架\(   \varepsilon ^{1}, \varepsilon ^{2}  \) .
        \item 计算 \(  \,\mathrm{d}   \varepsilon ^{1},\,\mathrm{d}  \varepsilon ^{2} \),带入Cartan第一结构方程 \[
        \,\mathrm{d}  \varepsilon ^{i}=  \varepsilon ^{j}\wedge  \omega _{j}^{i}
        \] 待定系数,或者通过缩并计算\(   \omega _{j}^{i}  \) 的分量. 
    \end{enumerate}
    
\end{method}

\begin{problem}
已知曲面的第一基本形式, 求 Gauss 曲率:
\begin{enumerate}
    \item $I = du du + u^2 dv dv;$
    \item $I = du du + \sin^2 u dv dv;$
\end{enumerate}
\end{problem}

\begin{proof}
    \begin{enumerate}
        \item 令 \(   \varepsilon ^{1}= \,\mathrm{d} u,  \varepsilon ^{2}=  u\,\mathrm{d} v   \).则 \[
        I=   \varepsilon ^{1}\otimes  \varepsilon ^{1}+  \varepsilon ^{2}\otimes  \varepsilon ^{2}
        \] 这表明 \(   \varepsilon ^{1}, \varepsilon ^{2}  \)是曲面的一组正交余标价.  由Cartan结构方程 \[
        0= \,\mathrm{d}  \varepsilon ^{1}=\varepsilon ^{1}  \wedge \omega _{1}^{1} + \varepsilon ^{2} \wedge  \omega _{1}^{2} = u\,\mathrm{d} v\wedge \omega _{1}^{2}
        \]\[
       \mathrm{d} u\wedge \,\mathrm{d} v= \,\mathrm{d}  \varepsilon ^{2}=   \varepsilon ^{j}\wedge\omega _{j}^{2}  =\varepsilon ^{1}  \wedge \omega _{2}^{1} =\,\mathrm{d} u\wedge   \omega _{2}^{1}
        \]可以得到 \[
         \omega _{1}^{2}= \,\mathrm{d} v
        \]由第二结构方程 \[
       \frac{1 }{u }K\,\mathrm{d} u\wedge \,\mathrm{d} v=   K \varepsilon ^{1}\wedge  \varepsilon ^{2}= \,\mathrm{d}  \omega _{2}^{1}=0
        \]得到 \(  K=0   \)
        \item 令 \(   \varepsilon ^{1}= \,\mathrm{d} u, \varepsilon ^{2}= \sin u\,\mathrm{d} v  \)  ,则 \[
        I=   \varepsilon ^{1}\otimes  \varepsilon ^{1}+  \varepsilon ^{2}\otimes  \varepsilon ^{2}
        \]表明 \(   \varepsilon ^{1}, \varepsilon ^{2}  \)是曲面的一组正交的余标架.由Cartan结构方程 \[
       0= \,\mathrm{d}  \varepsilon ^{1}=  \varepsilon ^{j}\wedge  \omega _{j}^{1}=  \varepsilon ^{2} \omega _{2}^{1}= \sin u \,\mathrm{d} v \wedge  \omega _{2}^{1}
        \] \[
        \cos u\,\mathrm{d} u\wedge \,\mathrm{d} v= \,\mathrm{d}  \varepsilon ^{2}=   \varepsilon ^{j} \omega _{j}^{2}=  \varepsilon ^{1}\wedge  \omega _{1}^{2}= \,\mathrm{d} u\wedge  \omega _{1}^{2}
        \]得到 \[
         \omega _{2}^{1}=- \cos u\,\mathrm{d} v
        \]从而由Cartan第二结构方程 \[
    \sin u K\,\mathrm{d} u\wedge \,\mathrm{d} v=     K \,  \varepsilon ^{1}\wedge  \varepsilon ^{2}=  \Omega _{2}^{1}= \,\mathrm{d}  \omega _{2}^{1}= \sin u\,\mathrm{d} u\wedge \,\mathrm{d} v
        \]得到 \(  K= 1  \) 
    \end{enumerate}
    

    \hfill $\square$
\end{proof}

\begin{problem}
 设两个曲面 $S$ 和 $\tilde{S}$ 的第一基本形式满足 $\operatorname{I} = \lambda \tilde{\operatorname{I}}$ ($\lambda > 0$, 常数), 证明:
$$
K = \frac{1}{\lambda} \tilde{K}.
$$
\end{problem}
\begin{proof}
    设 \(  S  \)的一组正交余标架是 \(   \varepsilon ^{1}, \varepsilon ^{2}  \),则 \[
    I= \left(  \varepsilon ^{1} \right)^{2}+ \left(  \varepsilon ^{2} \right)^{2}  
    \]则 \[
    \tilde{I}= \left( \sqrt{ \frac{1}{\lambda} }  \varepsilon ^{1} \right)^{2}+  \left( \sqrt{ \frac{1}{\lambda} }  \varepsilon ^{2} \right)^{2}  
    \]  这表明 \(  \tilde{\varepsilon} ^{1}: =  \sqrt{ \frac{1}{\lambda} } \varepsilon ^{1}  \), \( \tilde{\varepsilon} ^{2}: =   \sqrt{ \frac{1}{\lambda} }  \varepsilon ^{2}  \)构成 \(  \tilde{S}  \)的一个正交余标架.   由Cartan第一结构方程 \[
    \,\mathrm{d}  \varepsilon ^{1}=  \varepsilon ^{j}\wedge  \omega _{j}^{1},\quad \,\mathrm{d}   \tilde{\varepsilon} ^{1}=  \tilde{\varepsilon} ^{j}\wedge  \tilde{\omega} _{j}^{1}
    \]将 \(   \tilde{\varepsilon} ^{1} = \sqrt{ \frac{1}{\lambda} } \varepsilon ^{1} \)带入后一个方程,得到 \[
    \,\mathrm{d}  \varepsilon ^{1}=  \varepsilon ^{j}\wedge  \tilde{\omega} _{j}^{1}
    \] 于是 \[
     \varepsilon ^{j}\wedge  \omega _{j}^{1}=  \varepsilon ^{j}\wedge  \tilde{\omega} _{j}^{1}
    \]类似地 \[
     \varepsilon ^{j}\wedge  \omega _{j}^{2}=  \varepsilon ^{j}\wedge   \tilde{\omega} _{j}^{2}
    \]利用正交标价的反对称性,展开得到 \[
     \varepsilon ^{2}\wedge  \omega _{2}^{1}=  \varepsilon ^{2}\wedge  \tilde{\omega} _{2}^{1},\quad  \varepsilon ^{1}\wedge  \omega _{1}^{2}=  \varepsilon ^{1}\wedge  \tilde{\omega} _{1}^{2}
    \]于是 \[
     \omega _{2}^{1}=  \tilde{\omega} _{2}^{1}
    \]进而 \[
    \,\mathrm{d}  \omega _{2}^{1}= \,\mathrm{d}   \tilde{\omega} _{2}^{1}
    \]由Cartan第二结构方程, \[
    K  \varepsilon ^{1}\wedge  \varepsilon ^{2}= \,\mathrm{d}  \omega _{2}^{1}= \,\mathrm{d}  \tilde{\omega} _{2}^{1} =  \tilde{K}  \tilde{\varepsilon} ^{1}\wedge  \tilde{\varepsilon} ^{2}= \tilde{K}  \frac{1}{\lambda}  \varepsilon ^{1}\wedge  \varepsilon ^{2}
    \]得到\[
    K= \tilde{K}\frac{1 }{ \lambda  } 
    \] 

    \hfill $\square$
\end{proof}
\end{document}