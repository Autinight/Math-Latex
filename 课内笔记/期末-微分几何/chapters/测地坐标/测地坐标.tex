\documentclass[../../main.tex]{subfiles}

\begin{document}

\chapter{ 测地坐标 }
\section{测地平行坐标}
\begin{definition}
    设 \(  S  \)是正则曲面, \(   \gamma \left( v \right)   \)是 \(  S  \)上的一条测地线, \(  v\in I= \left( a,b \right)   \).
    对于任意的 \(  v\in I  \),令:
    \begin{enumerate}
        \item  \(  n\left( v \right)   \)是 \(  S  \)在 \(   \gamma \left( v \right)   \)处的单位法向量.       
        \item \(  w\left( v \right)= c^{\prime} \left( v \right)\times n\left( v \right)     \)是切平面上与 \(  c^{\prime} \left( v \right)   \)正交的向量.   
    \end{enumerate}
     则存在 \(   \delta  > 0  \),使得下面的映射是坐标空间到 \(  S  \)上一开子集的微分同胚 \[
      \sigma \left( u,v \right)= \exp _{ \gamma \left( v \right) }\left( u\cdot w\left( v \right)  \right)  
     \]  坐标空间为 \[
     \left\{ \left( u,v \right)\in \mathbb{R} ^{2}:- \delta  < u<  \delta  ,v\in I  \right\}
     \]
\end{definition}
\section{测地极坐标}
\begin{definition}
    设 \(  S  \)是正则曲面, \(  p \in S  \).令 \(   \varepsilon > 0  \)是 \(  p  \)的一个法邻域的半径. \(  \left\{ e_1,e_2 \right\}  \)是 \(  T_{p}S  \)的一组正交基.
    定义 \(  p  \)处的法极坐标为 \[
     \sigma \left( r, \theta  \right) =  \exp _{p}\left( \left( r\cos  \theta  \right)  e_1+ \left( r\sin  \theta  \right)e_2 \right)  
    \]   坐标空间为 \[
    \left\{ \left( r, \theta  \right)\in \mathbb{R} ^{2}: 0< r<  \varepsilon ,0<  \theta < 2\pi   \right\}
    \]
\end{definition}

\begin{lemma}{Gauss引理}
    设 \(  S  \)是正则曲面, \(  p \in S  \),\(  p  \)处法极坐标的局部第一基本形式形如 \[
    \mathcal{F}_{1}= \,\mathrm{d} r^{2}+ G\,\mathrm{d}  \theta ^{2}
    \]   
\end{lemma}

\begin{proof}
    固定 \(  0<  \theta  < 2\pi   \). 记 \(   \gamma _{v}\left( t \right)   \)为 \(  p  \)处以 \(  v := \cos  \theta_0e_1+ \sin  \theta e_2 \)为初速度的弧长参数测地线,则 \[
     \gamma _{v}\left( r \right)=  \gamma _{rv}\left( 1 \right)= \exp _{p}\left( rv \right)=  \sigma \left( r, \theta_0 \right)   
    \]   则 \(   \sigma _{r}\left( r, \theta   \right)=  \gamma ^{\prime} _{v}\left( r \right)    \),\[
    \left< \sigma _{r}, \sigma _{r} \right>= \left< \gamma _{v}^{\prime} , \gamma _{v}^{\prime}  \right>= 1
    \]这表明 \(  E= 1  \).
    
    \[
   \begin{aligned}
    F_{r}&=  \frac{\partial }{\partial r}\left< \sigma _{ \theta }, \sigma _{r} \right>\\ 
     &=  \left< \sigma _{ \theta r}, \sigma _{r} \right>+ \left< \sigma _{ \theta }, \sigma _{rr} \right>
   \end{aligned}
    \]其中 \[
     \sigma _{rr}=  \gamma _{v}^{\prime \prime} = 0
    \]于是 \[
    F_{r}= \left< \sigma _{ \theta r}, \sigma _{r} \right>= \frac{1 }{2 }\frac{\partial }{\partial  \theta }\left<  \sigma _{r}, \sigma _{r}\right>= \frac{1}{2}E_{ \theta }= 0
    \]依旧固定 \(  0<  \theta < 2\pi   \),则  \[
    \lim_{r\to 0} \sigma _{ \theta }\left( r, \theta  \right) = 0
    \]故 \[
    \lim_{r\to 0}F=  \lim_{r\to 0}\left< \sigma _{r}, \sigma _{ \theta }  \right>= 0 
    \]于是对于所有的 \(  r, \theta   \), \(  F\left( r, \theta  \right)= 0   \).  

    \hfill $\square$
\end{proof}

\begin{theorem}{\(  G\left( r, \theta  \right)   \)的几何意义 }
    设 \(  S  \)是正则曲面, \(  p \in S  \) ,  \(  p  \) 处极坐标的度量为 \[
   g\left( r, \theta  \right) = \,\mathrm{d} r^{2}+ G\left( r, \theta  \right)\,\mathrm{d}  \theta 
    \]   则  \begin{enumerate}
        \item 测地圆 \(  r= r_0  \)的弧长形式为 \[
    \,\mathrm{d} s= \sqrt{G\left( r, \theta  \right) }\,\mathrm{d}  \theta 
    \] 进而测地圆的周长为 \[
    L\left( r_0 \right)= \int_{0}^{2\pi }\sqrt{G\left( r_0, \theta  \right) }\,\mathrm{d}  \theta  
    \]
    \item 面积元 形式 \(  \,\mathrm{d} A  \)为 \[
    \,\mathrm{d} A= \sqrt{G\left( r, \theta  \right) }\,\mathrm{d} r\,\mathrm{d}  \theta 
    \] 
    \end{enumerate}
    
\end{theorem}
\begin{proof}
    \begin{enumerate}
        \item 设 \[
     \sigma \left( r, \theta  \right)= \exp _{p}\left( \left( r\cos  \theta  \right)e_1+ \left( r\sin  \theta  \right)e_2   \right)  
    \]是测地极坐标映射.则 \[
     \gamma \left(  \theta  \right): =   \sigma \left( r_0, \theta  \right)
    \]构成 \(  r= r_0  \)的一个坐标表示. \[
     \gamma ^{\prime} \left(  \theta  \right)=\left.  \partial _{ \theta } \right|_{\left( r_0, \theta  \right) }
    \] 于是 \[
    \,\mathrm{d} s= \left|  \gamma ^{\prime} \left(  \theta  \right)  \right|\,\mathrm{d}  \theta = \sqrt{\left<\left.  \partial _{ \theta } \right|_{\left( r_0, \theta  \right) },\left.  \partial _{ \theta } \right|_{\left( r_0, \theta  \right) } \right>}\,\mathrm{d}  \theta = \sqrt{G\left( r_0, \theta  \right) }\,\mathrm{d}  \theta  
    \]

    \item 体积形式为 \[
    \,\mathrm{d} V= \sqrt{\det g_{ij}}\,\mathrm{d} x^{1}\wedge \cdots \,\mathrm{d} x^{n}
    \]面积形式即为2维的体积形式,这里 \(  \det g_{ij}= G  \) 
    \end{enumerate}
    
    \hfill $\square$
\end{proof}

\begin{theorem}{\(  G  \)的极限行为  }
     设 \(  S  \)是正则曲面, \(  p \in S  \) ,  \(  p  \) 处极坐标的度量为 \[
   g\left( r, \theta  \right) = \,\mathrm{d} r^{2}+ G\left( r, \theta  \right)\,\mathrm{d}  \theta 
    \] 
    则
    \begin{enumerate}
        \item  \[
        \lim_{r\to 0}G\left( r, \theta  \right)= 0 
        \]
        \item \[
        \lim_{r\to 0} \frac{\sqrt{G\left( r, \theta  \right) } }{r }= 1 
        \]
        \item  \[
        \sqrt{G\left( r, \theta  \right) }= r-\frac{K\left( p \right)  }{6 }r^{3}+ O\left( r^{4} \right)  
        \]特别地,当 \(  K\equiv  1  \)时, \(  \sqrt{G}= \sin\left( r \right)    \) ,是单位球面的情况. 
    \end{enumerate}
    
\end{theorem}
\begin{proof}
    \begin{enumerate}
        \item 参数映射是 \[
     \sigma \left( r, \theta  \right) =  \exp _{p}\left( \left( r\cos  \theta  \right)  e_1+ \left( r\sin  \theta  \right)e_2 \right)  
    \] 则 \[
    \begin{aligned}
    \frac{\partial }{\partial  \theta }&=   \partial _{ \theta } \sigma \left( r, \theta  \right)  \\ 
     &= \left( \,\mathrm{d} \exp  \right)_{ ru\left(  \theta  \right)  }\left( -r\sin  \theta e_1+ r\cos  \theta e_2 \right)  
    \end{aligned}
    \]其中 \(  u\left(  \theta  \right)= \cos  \theta e_1+ \sin  \theta e_2   \),则 \[
    \lim_{r\to 0} \left( \,\mathrm{d} \exp  \right)_{r  u\left(  \theta  \right) }= \left( \,\mathrm{d} \exp  \right)_{0}= \operatorname{Id}_{T_{p}S}  
    \] 从而 \[
    \lim_{r\to 0} \frac{\partial }{\partial  \theta }= \operatorname{Id}_{T_{p}S} \lim_{r\to 0}\left( -r\sin  \theta e_1+ r\cos  \theta e_2 \right)= 0 
    \]这表明 \(  \lim_{r\to 0}G\left( r, \theta  \right)= 0   \) 
    \item 直接证明3.正交标价的方法给出 \[
    K= -\frac{1 }{\sqrt{G} }\frac{ \partial ^{2}\sqrt{G} }{ \partial r^{2} }  
    \]由于 \(  \sqrt{G}  \)的展开式只有奇数次幂,设 \(  \sqrt{G}= Ar+ Br^{3}+ Cr^{5}+ O\left( r^{7} \right)   \)带入计算即可.  
    \end{enumerate}
    

    \hfill $\square$
\end{proof}
\section{常曲率度量的极分解}

\begin{lemma}
    方程 \[
    u^{\prime \prime} \left( t \right)+ c u\left( t \right)= 0,\quad u\left( 0 \right)= 0   
    \]的解空间是函数
    \[
    s_{c}\left( t \right)= \begin{cases} t,&c= 0\\ 
     R\sin \frac{t }{R },& c =  \frac{1 }{R^{2} }> 0\\ 
      R\sinh  \frac{t }{R },&c =-\frac{1 }{R^{2} }< 0     \end{cases}  
    \]张成的一维线性子空间.
\end{lemma}

\begin{proposition}{常曲率空间的Jacobi场}
    设 \(  \left( M,g \right)   \) 是有常曲率 \(  c  \)的Riemann流形, \(   \gamma   \)是 \(  M  \)上的单位速度测地线.   则沿 \(   \gamma   \)法向,且在 \(  t= 0  \)处消失的 Jacobi场具有 以下形式: \[
    J\left( t \right)= k s_{c}\left( t \right)  E\left( t \right) 
    \]其中 \(  E  \)是任意沿 \(   \gamma   \)平行的单位法向量场, \(  k  \)是任意常数.这样的Jacobi场的初值是 \[
    D_{t}J\left( 0 \right)= kE\left( 0 \right)  
    \]范数为 \[
    \left| J\left( t \right)  \right|= \left| s_{c}\left( t \right)  \right|\left| D_{t}J\left( 0 \right)  \right|   
    \]   
\end{proposition}
\begin{definition}
    令 \(  \pi : \mathbb{R} ^{n}\setminus \left\{ 0 \right\}\to \mathbb{S}^{n-1}  \)是径向投影 \[
    \pi \left( x \right)= \frac{x }{\left| x \right|  }  
    \] 定义 \(  \mathbb{R} ^{n}\setminus \left\{ 0 \right\}  \)上的一个对称2-张量 \[
    \hat{g}= \pi ^{*}\overset{\scriptstyle\circ}{g}
    \] 其中 \(  \overset{\scriptstyle\circ}{g}  \)是半径为1的 \(  \mathbb{S}^{n-1}  \)上 的圆度量. 
\end{definition}
\begin{note}
     \(  \hat{g}  \)只保留角度信息. 
\end{note}

\begin{lemma}{欧氏度量的极分解}
    在 \(  \mathbb{R} ^{n}\setminus \left\{ 0 \right\}  \)上,欧式度量 \(  \bar{g}  \)分解为 \[
    \bar{g}= \,\mathrm{d} r^{2}+ r^{2} \hat{g}
    \]其中 \(  r\left( x \right)= \left| x \right|    \)是到原点的欧氏距离.   
\end{lemma}
\begin{proof}
    \[
    \Phi :\mathbb{R} ^{+ }\times _{\rho }\mathbb{S}^{n-1}\to \mathbb{R} ^{n}\setminus \left\{ 0 \right\},\quad \Phi \left( \rho , \omega  \right)= \rho  \omega  
    \]给出warped积空间 \(  \mathbb{R} ^{+ }\times _{\rho }\mathbb{S}^{n-1}  \)到欧氏子空间 \(  \mathbb{R} ^{n}\setminus \left\{ 0 \right\}  \)的等距同构. \(  \Phi ^{-1} \left( x \right)= \left( r\left( x \right),\pi \left( x \right)   \right)    \),于是 \[
    \bar{g}= \left( \Phi ^{-1}  \right)^{*}\left( \,\mathrm{d} \rho ^{2}\oplus \rho ^{2}\overset{\scriptstyle\circ}{g} \right)= \,\mathrm{d} r^{2}+ r^{2}\hat{g}  
    \]   
    \hfill $\square$
\end{proof}

\begin{theorem}{法坐标上的常曲率度量}
    设 \(  \left( M,g \right)   \)是具有常值截面曲率的黎曼流形.给定 \(  p \in M  \),令 \(  \left( x^{i} \right)   \)是\(  p  \)的法邻域 \(  U  \)上的一个法坐标. \(  r  \)是 \(  U  \)上的径向距离函数. \(  \hat{g}  \)是上面定义的对称2-张量.则在 \(  U\setminus \left\{ p \right\}  \)上, 度量 \(  g  \)可以进行考虑曲率修正的极分解 \[
    g =  \,\mathrm{d} r^{2}+ s_{c}\left( r \right)^{2}\hat{g} 
    \]          
    
\end{theorem}
\begin{proof}
    令 \(  g_{c}  \)是右侧形式的度量, \(  \bar{g}  \)是欧式度量.   
    \(  g  \), \(  \bar{g}  \)和 \(  g_{c}  \)均在 \(   \partial _{r}  \)    上产生相同的作用.因此只需要证明对于每个对于任意的水平集 \(  r= b  \),以及任意相切于该水平集的切向量 \(  w  \),都有 \(  g\left( w,w \right)= g_{c}\left( w,w \right)    \).首先根据定义 \[
    g_{c}\left( w,w \right)= s_{c}\left( r^{2} \right)  \hat{g}\left( w,w \right)= \frac{s_{c}\left( b \right)^{2}  }{b^{2} }\bar{g}\left( w,w \right)   
    \]   

    令 \(  q \in U\setminus \left\{ p \right\}  \),\(  w\in T_{p}M  \),\(  w  \)相切于包含了 \(  q  \)的 \(  r  \)-水平集. \(  b = d _{g}\left( p,q \right)   \).由于 \(  g  \)只有在原点是我们熟知的,与 \(  \bar{g}  \)相等. 上面的等式告诉我们联系 \(  g  \)和 \(  g_{c}  \)相当于联系 \(  \bar{g}  \)和 \( g  \),我们通过沿 \(  p  \)到 \(  q  \)的径向测地线的Jacobi场将 \(  g  \)在 \(  p  \)的取值和 \(  q  \)的取值联系起来   .     设 \(   \gamma :[0,b]\to U  \)是单位参数化的\(  p  \)到 \(  q  \)的  径向测地线.令 \(  J \in \mathfrak{X}\left(  \gamma  \right)   \)是沿 \(   \gamma   \)的测地线,满足 \[
    J\left( t \right)= \frac{t }{b }w^{i}\left.  \partial _{i} \right|_{ \gamma \left( t \right) }  
    \].则 \(  D_{t}J\left( 0 \right)= \left( \frac{1}{b} \right)w^{i}\left.  \partial _{i} \right|_{p}    \)   . \(  J\left( 0 \right)= 0, J\left( b \right)= w    \)         在两点处与 \(   \gamma ^{\prime}   \)正交,从而是一个法向量场.于是有范数的关系 \[
    \begin{aligned}
        \left| w \right|_{g}^{2}&= \left| J\left( b \right)  \right|_{g}^{2}= s_{c}\left( b \right)^{2}\left| D_{t}J\left( 0 \right)  \right|_{g}^{2}  \\ 
         &= s_{c}\left( b \right)^{2} \frac{1 }{b^{2} }\left| w^{i} \left.  \partial _{i} \right|_{p} \right|_{g}^{2}= s_{c}\left( b \right)^{2}\frac{1 }{b^{2} }\left| w \right|_{\bar{g}}^{2}= \left| w \right|_{g_{c}}          ^{2}
    \end{aligned}
    \] 

    \hfill $\square$
\end{proof}

\begin{corollary}{常曲率度量的局部唯一性}
    设 \(  \left( M,g \right)   \)和 \(  \left( \tilde{M},\tilde{g} \right)   \)是有着相同维数,且具有相同常曲率 \(  c  \)的Riemann流形.则对于每个 \(  p \in M  \),\(  \tilde{p}\in \tilde{M}  \)     ,存在 \(  p  \)的邻域 \(  U  \)和 \(  \tilde{p}  \)的邻域 \(  \tilde{U}  \),使得它们之间存在等距同构 \(   \varphi : U\to \tilde{U}  \).     
\end{corollary}

\begin{proof}
    常曲率度量的这种与法坐标选取无关的显式的一致表达,就能给出一个坐标的等同就是一个等距同构.即令 \(  \psi :U\to B_{ \varepsilon }\left( 0 \right)\subseteq \mathbb{R} ^{n}   \)和 \(  \tilde{\psi} :\tilde{U}\to B_{ \varepsilon }\left( 0 \right)   \subseteq \mathbb{R} ^{n}\)是法坐标映射, \(  \tilde{\Psi} ^{-1} \circ \Psi   \)就是所需的等距同构:原点以外形式上一致,原点上都是单位阵.

    \hfill $\square$
\end{proof}

\end{document}