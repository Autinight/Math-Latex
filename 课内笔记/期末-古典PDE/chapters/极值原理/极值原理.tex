\documentclass[../../main.tex]{subfiles}

\begin{document}

\chapter{ 极值原理 }

\section{调和函数的极值原理}

\begin{definition}{调和函数}
    设 \(   \Omega   \)是 \(  \mathbb{R} ^{n}  \)上的一个开集.设 \(  u \in C^{2}\left(  \Omega  \right)   \).若 \(   \Delta u= 0  \)在 \(   \Omega   \)上成立,则称 \(  u  \)是 \(   \Omega   \)上的调和函数.       
\end{definition}


\begin{theorem}{平均值性质}
    若 \(  u  \)是 \(   \Omega   \)上的调和函数 ,\(  B_{r}\left( x_0 \right)\subseteq  \Omega    \).则
    \begin{enumerate}
        \item \[
        u\left( x_0 \right)= \frac{1 }{\mathrm{vol}\left( B_{r}\left( x_0 \right)  \right)  }\int_{B_{r}\left( x_0 \right) }u\left( x \right)\,\mathrm{d} x   
        \]
        \item \[
        u\left( x_0 \right)= \frac{1 }{\mathrm{area}\left(  \partial B_{r}\left( x_0 \right)  \right)  }\int_{ \partial B_{r}\left( x_0 \right) }u\left( x \right)\,\mathrm{d} S   
        \]
    \end{enumerate}
       
\end{theorem}


\section{调和函数的极值原理}

\begin{theorem}{弱极值原理}
    设 \(   \Omega   \)是 \(  \mathbb{R} ^{n}  \)上的一个有界开集.若 \(  u  \)是 \(   \Omega   \)上的调和函数,且在 \(   \bar{\Omega}  \)上连续.则
    \begin{enumerate}
        \item \(  u  \)在 \(   \bar{\Omega}  \)上的最大值在边界 \(   \partial  \Omega   \)上取得: \[
        \max _{x\in  \bar{\Omega}}u\left( x \right)= \max _{x\in  \partial  \Omega }u\left( x \right)  
        \]
        \item \(  u  \)在 \(   \bar{\Omega}  \)上的最小值在边界 \(   \partial  \Omega   \)上取得 :  \[
        \min _{x\in  \bar{\Omega}}u\left( x \right)= \min _{x\in  \partial  \Omega }u\left( x \right)  
        \]    
    \end{enumerate}
         
\end{theorem}


\begin{theorem}{强极值原理}
    设 \(   \Omega   \)是 \(  \mathbb{R} ^{n}  \)中的一个连通开集.若 \(  u  \)是 \(   \Omega   \)上的调和函数.若以下成立其一 :
    \begin{enumerate}
        \item  \(  u  \)在 \(   \Omega   \)的内部某点 \(  x_0  \)取得局部最大值
        \item \(  u  \)在 \(   \Omega   \)的内部某点 \(  x_0  \)取得局部最小值.   
    \end{enumerate}
    ,则 \(  u  \)在 整个\(   \Omega   \)上是常函数.         
\end{theorem}

\section{热方程的极值原理}
本节中,考虑区域 \(   \Omega \subseteq \mathbb{R} ^{n}  \)上,时间 \(  t\in (0,T]  \)内的热方程 \[
u_{t}- \Delta u= 0,\quad \left( x,t \right)\in  \Omega \times (0,T] 
\]  

\begin{definition}{抛物型边界}
    对于热方程,定义其抛物型边界(Parabolic Boundary) \(   \partial _{p} \Omega _{T}  \)为 \[
     \partial _{p} \Omega _{T}= \left(  \bar{\Omega}\times \left\{ 0 \right\} \right)\cup \left(  \partial  \Omega \times \left[ 0,T \right]  \right)  
    \] 
\end{definition}

\begin{theorem}{弱最大值原理}
    设 \(  u\left( x,t \right)\in C^{2}\left(  \Omega _{T} \right)\cap C_0\left( \overline{ \Omega _{T}} \right)     \) 是在 \(   \Omega _{T}  \)上满足 \(  u_{t}- \Delta u= 0  \)的一个经典解.那么 \(  u  \)在 \(   \bar{\Omega}_{T}  \)上的最大值和最小值一定在抛物型边界 \(   \partial _{p} \Omega _{T}  \)上取得. 即: \[
    \max _{\left( x,t \right)\in   \bar{\Omega_{T}} }u\left( x,t \right)= \max _{\left( x,t \right)\in  \partial _{p} \Omega _{T} }u\left( x,t \right)  
    \]    
\end{theorem}


\begin{theorem}{强最大值原理}
  设 \(  u\left( x,t \right)\in C^{2}\left(  \Omega _{T} \right)\cap C_0\left( \overline{ \Omega _{T}} \right)     \) 是在 \(   \Omega _{T}  \)上满足 \(  u_{t}- \Delta u= 0  \)的一个经典解. 如果 \(  u  \)在 \(   \Omega _{T}  \) 的内部某个点 \(  \left( x_0,t_0 \right)   \)处达到 \(  \overline{Q_{T}}  \)     上的最大值,那么 \(  u  \)在区域 \(   \bar{\Omega}\times \left[ 0,T \right]   \)上必须是常数.  
\end{theorem}


\begin{problemsec}




    \begin{problem}
\begin{enumerate}
    \item 设 $\Omega \subset \mathbb{R}^n$ 为有界区域, 记 $\Omega' = \mathbb{R}^n \setminus \overline{\Omega}$. 考虑如下调和方程的 Dirichlet 外问题:
    \[ \begin{cases} -\Delta u = 0, & x \in \Omega' \\ u|_{\partial \Omega} = \varphi(x) \\ \lim_{|x| \to +\infty} u(x) = 0. \end{cases} \]
    利用调和方程的极值原理, 证明上述外问题的经典解是唯一的.
    \item 并举例说明, 在没有 $\lim_{|x| \to +\infty} u(x) = 0$ 的条件下经典解可以是不唯一的.

\end{enumerate}
\end{problem}

\begin{proof}
    \begin{enumerate}
        \item 设 \(  u_1,u_2  \)是外问题的两个经典解,令 \(  w= u_1-u_2  \),则 \(  w  \)是满足以下性质 \[
        \begin{cases} - \Delta w= 0,\\ 
         w|_{ \partial  \Omega }=0,\\ 
          \lim_{\left| x \right|\to + \infty }w\left( x \right)= 0   \end{cases} 
        \]   对于任意的 \(  R> 0  \),使得 \(   \bar{\Omega}\subseteq B_{R}\left( 0 \right)   \),定义 \(  D_{R}: =  B_{R}\left( 0 \right)\setminus  \bar{\Omega}   \)  ,则 \(  D_{R}  \)是 \(  \mathbb{R} ^{n}  \)上的有界开集,并且  \(   \partial D_{R}=  \partial  \Omega \cup   \partial B_{R}\left( 0 \right)   \)   . 
        
        由于 \(  \lim_{\left| x \right|\to + \infty }w\left( x \right)   \),对于任意的 \(   \varepsilon > 0  \),存在 \(  R\left(  \varepsilon  \right)> 0   \),使得 \(   \bar{\Omega}\subseteq B_{R }\left( 0 \right)   \)且 \(  \sup _{x\in  \partial B_{R }\left( 0 \right) }\left| w\left( x \right)  \right|<  \varepsilon    \) 对于所有的 \(  R> R\left(  \varepsilon  \right)   \)成立     .又 \(  w|_{ \partial  \Omega }= 0  \) 由极值原理, \[
        - \varepsilon < \min _{x\in D_{R}}w\left( x\right)\le w\left( x \right)\le  \max _{x\in D_{R }}w\left( x \right)<  \varepsilon   , \quad\forall R> R\left(  \varepsilon  \right), \forall x\in  \partial D_{R},
        \]对于 \(   \Omega ^{\prime}   \)中任意一点 \(  x_0  \),可以选取足够大的 \(  R  \)(比如 \(  R = \max \left( R\left(  \varepsilon  \right),\left| x_0 \right|+ 1   \right)   \) ),使得 \[
        \left| w\left( x_0 \right)  \right|\le  \varepsilon  
        \]  由于 \(   \varepsilon   \)是任取的,   \(  w\left( x_0 \right)= 0   \).这表明 \(  w\equiv 0  \). 经典解唯一.   
        \item 考虑\(  \mathbb{R} ^{3}  \)上的 外问题 \[
        \begin{cases} - \Delta u= 0,\\ 
         u|_{ \partial B_{1}\left( 0 \right) }= 0 \\ 
          \lim_{\left| x \right|\to \infty }u\left( x \right)= 0 \end{cases} 
        \]令 \[
        u\left( x \right)= 1-\frac{1 }{\left| x \right|  }  
        \],则 \[
         \Delta u=   \Delta \left( -\frac{1 }{r }  \right)= \frac{1 }{r^{2} } \partial _{r}\left( r^{2} \partial _{r}\left( -\frac{1 }{r }  \right)  \right)=    0,\quad r> 0
        \]于是 \[
        u\left( x \right)= 1-\frac{1 }{\left| x \right|  }  
        \]是外问题的一个解,但是 \(  u \equiv 0  \)也是一个解,这就说明了外问题不具有唯一性. 
    \end{enumerate}
    \hfill $\square$
\end{proof}


\end{problemsec}
\end{document}