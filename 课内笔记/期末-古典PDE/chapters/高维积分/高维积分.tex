\documentclass[../../main.tex]{subfiles}

\begin{document}

\chapter{ 高维积分 }
\section{分部积分}
\begin{theorem}{流形上的Stokes公式}
    对于 \(  n  \)-维定向带边流形 \(  M  \),以及任意的 \(  \left( n-1 \right)   \)-形式 \(   \omega   \),都有 \[
    \int_{M}\,\mathrm{d}  \omega = \int_{ \partial M} \omega 
    \]   
\end{theorem}

\begin{lemma}
    设 \(  \left( M,g \right)   \)是 \(  n  \)-维定向带边Riemann流形. \(   \,\mathrm{d} V_{g}\)和 \(   \,\mathrm{d} S_{g} \)分别是 \(  M  \)和 \(   \partial M  \)的Riemann体积形式, \(  N  \)是 \(   \partial M  \)上的单位外法向量场.则 \[
    \mathrm{d} S_{g}=  \left. i_{N}  \,\mathrm{d} V_{g}\right|_{ \partial M}
    \] 

    具体地,在给定局部标架 \(   x^1,\cdots,x^n   \)下,设 \(  N= N^{i}\frac{\partial }{\partial x^{i}}  \),则 \[
     \,\mathrm{d} V_{g}= \sqrt{g}\,\mathrm{d} x^{1}\wedge \cdots \wedge \,\mathrm{d} x^{n}
    \]   \[
   \begin{aligned}
   \mathrm{d} S_{g}&= \sqrt{g}\left( \,\mathrm{d} x^{1}\wedge \cdots \wedge \,\mathrm{d} x^{n} \right)\left( N^{i}\frac{\partial }{\partial x^{i}} \right)\\ 
     & = \sum _{i= 1}^{n}\sqrt{g}N^{i}\left( -1 \right)^{i-1} \,\mathrm{d} x^{1}\wedge \cdots \wedge \widehat{\,\mathrm{d} x^{i}}\wedge \cdots \wedge \,\mathrm{d} x^{n}   
   \end{aligned}
    \]
\end{lemma}

\begin{lemma}
    设 \(  \left( M,g \right)   \)是 \(  n  \)-维定向带边Riemann流形. \(   \,\mathrm{d} V_{g}\)和 \(   \,\mathrm{d} S_{g} \)分别是 \(  M  \)和 \(   \partial M  \)的Riemann体积形式, \(  N  \)是 \(   \partial M  \)上的单位外法向量场.则对于任意 \(  M  \)上的向量场 \(  X  \),在 \(   \partial M  \)上有 以下成立 \[
    i_{X} \,\mathrm{d} V_{g}= g\left( X,N \right)\,\mathrm{d} S_{g} 
    \]  
\end{lemma}
\begin{proof}
令 \[
    X^{\top}= X- g\left( X,N \right)N 
    \]则 \(  X^{\top}  \)是 \(   \partial M  \)上的一个切向量场.设 \(   v_1,\cdots,v_{n-1}  \)是 \(   \partial M  \)上的 \(  \left( n-1 \right)   \)个向量场,则 \[
    \begin{aligned}
    i_{X}\,\mathrm{d} V_{g}\left(  v_1,\cdots,v_{n-1} \right)&= \,\mathrm{d} V_{g}\left( X,v_1,\cdots ,v_{n-1} \right) \\ 
     &= \,\mathrm{d} V_{g}\left( X^{\top}+ g\left( X,N \right)N,v_1,\cdots ,v_{n-1}  \right)\\ 
      &=   g\left( X,N \right)\,\mathrm{d} V_{g}\left( N,v_1,\cdots ,v_{n-1} \right)+ \,\mathrm{d} V_{g}\left( X^{\top},v_1,\cdots ,v_{n-1} \right).   
    \end{aligned} 
    \]   其中 \[
    g\left( X,N \right)\,\mathrm{d} V_{g}\left( N, v_1,\cdots,v_{n-1} \right)= g\left( X,N \right)\mathrm{d} S_{g}\left(  v_1,\cdots,v_{n-1}\right)    
    \]此外, \(  X^{\top}, v_1,\cdots,v_{n-1}  \)是 \(  \left( n-1 \right)   \)-流形 \(   \partial M  \)上的 \(  n  \)个光滑向量场,其必 \(  C^{\infty}\left(  \partial M \right)   \)-线性相关.从而 \[
    \,\mathrm{d} V_{g}\left( X^{\perp}, v_1,\cdots,v_{n-1} \right)= 0 
    \]     于是 \[
    i_{X}\,\mathrm{d} V_{g}\left(  v_1,\cdots,v_{n-1} \right)= g\left( X,N \right)\,\mathrm{d} S_{g}\left(  v_1,\cdots,v_{n-1} \right)   
    \]即 \[
    i_{X}\,\mathrm{d} V_{g}= g\left( X,N \right)\,\mathrm{d} S_{g} 
    \]
    \hfill $\square$
\end{proof}


\begin{definition}{Lie导数}
    设 \(  M  \)是光滑流形, 设 \(  X  \)是 \(  M  \)上的一个向量场, \(  T  \)上 \(  M  \)上的一个张量场.按一以下方式定义 \(  T  \)沿着 \(  X  \)的Lie导数 \(  \mathcal{L}_{X}T  \)
    \begin{enumerate}
        \item 若\(  T= f\in C^{\infty}\left( M \right)   \),定义 \[
        \mathcal{L}_{X}T= \mathcal{L}_{X}f= X\left( f \right) 
        \]
        \item 若 \(  T=  Y \in \mathfrak{X}\left( M \right)   \),定义 \[
        \mathcal{L}_{X}T= \mathcal{L}_{X}Y: =  \left[ X,Y \right] 
        \]  
        \item 若 \(  T=  \omega \in  \Omega \left( M \right)   \),定义 \[
        \mathcal{L}_{X}T= \mathcal{L}_{X} \omega = \,\mathrm{d} \left( i_{X} \omega  \right) + i_{X}\,\mathrm{d}  \omega 
        \] 这个等式被称为是Cartan魔术公式.
        \item 对于两个张量场\(  S,T  \) ,  \[
        \mathcal{L}_{X}\left( S\otimes T \right)= \left( \mathcal{L}_{X}S \right)\otimes T+ S\otimes \left( \mathcal{L}_{X}T \right)   
        \]因此递归地定义出 任意  \(  \left( k,l \right)   \)-张量场沿着 \(  X  \)的Lie导数.  
    \end{enumerate}
           
\end{definition}

\begin{definition}{散度}
      对于光滑流形 \(  M  \).设 \(  X  \)是 \(  M  \)上的光滑向量场, \(   \Omega   \)是 \(  M  \)上的一个体积形式,定义 \(  X  \)关于 \(   \Omega   \)的散度,为一个     光滑函数 \(  \mathrm{div}_{ \Omega }\left( X \right)   \) 使得 \[
      \mathcal{L}_{X} \Omega = \left( \mathrm{div}_{ \Omega }\left( X \right)  \right) \Omega  
      \]

      特别地,在标准欧式空间 \(  \mathbb{R} ^{n}  \),向量值函数 \(  \mathbf{F}  = \left( F^{1},\cdots ,F^{n} \right) \)关于标准体积形式 \(  \,\mathrm{d} x^{1}\wedge \cdots \,\mathrm{d} x^{n}  \) 的散度为   \[
      \mathrm{div}\left( F \right)= \sum _{i= 1}^{n}\frac{\partial F^{i}}{\partial x^{i}} 
      \]
\end{definition}
\begin{remark}
    \[
    \mathcal{L}_{X}\left( \,\mathrm{d} x \right)= \,\mathrm{d} \left( i_{\mathbf{F}}\,\mathrm{d} x \right)+ i_{\mathbf{F}}\,\mathrm{d} \left( \,\mathrm{d} x \right)= \,\mathrm{d} \left( \sum _{i= 1}^{n}\left( -1 \right)^{j-1}F^{j} \hat{\,\mathrm{d} x^{j}}  \right)= \sum _{j= 1}^{n}\frac{\partial F^{j}}{\partial x^{j}}\,\mathrm{d} x  
    \]
\end{remark}

\begin{theorem}{散度定理}
    设 \(  \left( M,g \right)   \)是 光滑可定向黎曼流形. \(  \mathbf{F}  \)是 \(  M  \)上的一个光滑向量场, \(  N  \)是 \(   \partial M  \)上的单位外法向量场.则 \[
    \int_{ \Omega } \mathrm{div}\left(  \mathbf{F} \right) \,\mathrm{d} V_{g}=  \int_{ \partial  \Omega }g\left( \mathbf{F},N \right)\,\mathrm{d} S_{g}  
    \]    
\end{theorem}

\begin{proof}
    由Stokes定理 \[
    \int_{M}\,\mathrm{d} \left( i _{\mathbf{F}}\,\mathrm{d} V _{g}\right)= \int_{ \partial M} i_{\mathbf{F}}\,\mathrm{d} V_{g} 
    \]其中 \[
    i_{\mathbf{F}}\,\mathrm{d} V_{g}= g\left( \mathbf{F},N \right)\,\mathrm{d} S_{g} 
    \]且由Cartan魔术公式, \[
     \mathrm{div}\left( \mathbf{F} \right)\,\mathrm{d} V_{g}= \mathcal{L}_{\mathbf{F}}\left( \,\mathrm{d} V _{g}\right) = \,\mathrm{d} \left( i_{\mathbf{F}}\,\mathrm{d} V_{g} \right)+ i_{\mathbf{F}}\,\mathrm{d} \left( \,\mathrm{d} V_{g} \right)= \,\mathrm{d} \left( i_{\mathbf{F}}\,\mathrm{d} V_{g} \right)   
    \]带入即得

    \hfill $\square$
\end{proof}
\begin{corollary}{Gauss-Green}
        设 \(   \Omega \subseteq \mathbb{R} ^{n},\left(  x_1,\cdots,x_n  \right)   \)是标准坐标,则 \[
    \int_{ \Omega }u_{x_{i}}\,\mathrm{d} x= \int_{ \partial  \Omega }u N ^{i}\,\mathrm{d} S
    \] 其中 \(  \,\mathrm{d} x= \,\mathrm{d} x^{1}\cdots \,\mathrm{d} x^{n}\); \(  i=  1,\cdots,n   \).  
\end{corollary}
\begin{proof}
 令 \(  \mathbf{F}= u\frac{\partial }{\partial x^{i}}  \).则 \[
 \mathrm{div}\left( \mathbf{F} \right) = u_{x^{i}}
 \]  \[
 \mathbf{F}\cdot N= uN^{i}
 \]由散度定理立即得到.
    \hfill $\square$
\end{proof}

\begin{lemma}
     设 \(  \left( M,g \right)   \)是 光滑可定向黎曼流形. \(  \mathbf{F}  \)是 \(  M  \)上的一个光滑向量场, \(  v  \)是 \(  M  \)上的 函数,则 \[
    \left( \mathrm{div}  \mathbf{F} \right)v= \mathrm{div}\left( v \mathbf{F} \right)-g\left( \operatorname{grad}\,v, \mathbf{F} \right)   
    \]  
\end{lemma}
\begin{proof}
    由 \[
\mathcal{L}_{v \mathbf{F}}\left( \mu  \right)= \,\mathrm{d} \left( i_{v \mathbf{F}}\mu  \right)  = \,\mathrm{d} \left( v i _{\mathbf{F}}\mu  \right)= \,\mathrm{d} v\wedge  i_{\mathbf{F}}\mu+ v \,\mathrm{d} \left( i _{\mathbf{F}}\mu  \right)
\]  其中,设 \(  \mu = \,\mathrm{d} x^{1}\wedge \cdots \,\mathrm{d} x^{n}  \),则 \[
\begin{aligned}
\left( \,\mathrm{d} v\wedge i_{\mathbf{F}}\mu  \right)\left( \frac{\partial }{\partial x^{1}},\cdots ,\frac{\partial }{\partial x^{n}} \right)&= \sum _{j= 1}^{n}  \left( -1 \right)^{j-1} \left( \,\mathrm{d} v \right)\left( \frac{\partial }{\partial x^{j}} \right)\mu \left( \mathbf{F}, \frac{\partial }{\partial x^{1}},\cdots ,\hat{\frac{\partial }{\partial x^{j}}},\cdots ,\frac{\partial }{\partial x^{n}} \right)    \\ 
 &= \sum _{j= 1}^{n}g\left( \operatorname{grad}\,v, \frac{\partial }{\partial x^{j}} \right)F^{j}= g\left( \operatorname{grad}\,v,\mathbf{F} \right)   
\end{aligned}
\]  即 \(  \left( \,\mathrm{d} v\wedge i_{\mathbf{F}}\mu  \right)   = g\left( \operatorname{grad}\,v,\mathbf{F} \right) \).此外 \(  v\,\mathrm{d} \left( i_{F}\mu  \right)\mu = v \mathcal{L}_{ \mathbf{F}}\left( \mu  \right)    \)  这表明 \[
\mathrm{div}\left( v \mathbf{F} \right)=  g\left( \operatorname{grad}\,v,\mathbf{F} \right) +  v \;\mathrm{div}\left( \mathbf{F} \right) 
\]

    \hfill $\square$
\end{proof}
\begin{theorem}{高维分部积分}
     设 \(  \left( M,g \right)   \)是 光滑可定向黎曼流形. \(  \mathbf{F}  \)是 \(  M  \)上的一个光滑向量场,  \(  v  \)是 \(  M  \)上的一个光滑函数.则 \[
     \int_{M}\left( \mathrm{div} \mathbf{F} \right)v \,\mathrm{d} V_{g}= \int_{ \partial M}v g\left( \mathbf{F},N \right)\,\mathrm{d} S_{g}- \int_{  M}g\left( \operatorname{grad}\,v, \mathbf{F} \right)\,\mathrm{d} V_{g}   
     \]  
\end{theorem}

\begin{proof}
    对引理的等式在 \(   \Omega   \)上积分,并利用散度定理立即得到. 

    \hfill $\square$
\end{proof}


\begin{corollary}{某一方向上的分部积分}
      设 \(   \Omega   \)是 \(  \mathbb{R} ^{n}  \)上的一个可定向嵌入Riemann超曲面, \(  \left(  x^1,\cdots,x^n  \right)   \)是 \(  \mathbb{R} ^{n}  \)的标准坐标  . 设 \(  u,v  \)是 \(   \Omega   \)上的光滑函数, \(  N= N^{j}\frac{\partial }{\partial x^{j}} \)是 \(   \partial  \Omega   \)上的单位外法向量场.    则  \[
      \int_{ \Omega } uv\,\mathrm{d} V=  \int_{ \partial  \Omega } uvN^{i}\,\mathrm{d} S- \int_{ \Omega } uv_{x_{i}}\,\mathrm{d} V
      \]
\end{corollary}
\begin{proof}
    上面的分部积分公式令 \(  \mathbf{F}= u \frac{\partial }{\partial x^{i}}  \) 即可.

    \hfill $\square$
\end{proof}

\begin{theorem}{Green公式}
    设 \(  u,v \in C^{2}\left( \overline{ \Omega } \right)   \),\(  \nu   \)是单位外法向量, 那么 
    \begin{enumerate}
        \item \[
        \int_{ \Omega } \Delta u\,\mathrm{d}x= \int_{ \partial  \Omega }\frac{\partial u}{\partial \nu }\,\mathrm{d} S
        \]\item  \[
        \int_{ \Omega }  \nabla v \cdot  \nabla u\,\mathrm{d} x= - \int_{ \Omega }u \Delta v\,\mathrm{d} x+  \int_{ \partial  \Omega }\frac{\partial v}{\partial \nu } u\,\mathrm{d} S  
        \]
        \item \[
        \int_{ \Omega }\left( u \Delta v-v  \Delta u \right)\,\mathrm{d} x= \int_{ \partial  \Omega }\left( u\frac{\partial v}{\partial \nu }-v \frac{\partial u}{\partial \nu } \right)\,\mathrm{d} S  
        \]其中2,3分别称为格林第一,二公式.
    \end{enumerate}
    
\end{theorem}

\begin{proof}
    \begin{enumerate}
        \item 在欧氏空间上, \(   \Delta u= \mathrm{div}\left(  \nabla u \right)   \), \(  \frac{\partial u}{\partial \nu }=  \nabla u\cdot  \nu   \)  .令 \(  F =   \nabla u  \),由散度定理, \[
        \int_{ \Omega } \Delta u\,\mathrm{d} x=  \int_{ \Omega } \mathrm{div}\left(  \nabla u \right)=  \int_{ \partial  \Omega } \nabla u\cdot  \nu \,\mathrm{d} S=  \int_{ \partial  \Omega } \frac{\partial u}{\partial \nu }\,\mathrm{d} S 
        \] 
        \item  令 \(  \mathbf{F}=  \nabla v  \),由高维分部积分 \[
        \int_{ \Omega } u  \,\Delta v\,\mathrm{d} x=  \int_{ \partial  \Omega } u\frac{\partial v}{\partial \nu }\,\mathrm{d} S- \int_{ \Omega }  \nabla v\cdot  \nabla u\,\mathrm{d} x
        \] 
        \item 由2. \[
        \int_{ \Omega }u \Delta v\,\mathrm{d} x=  \int_{ \partial  \Omega } \frac{\partial v}{\partial \nu }u \,\mathrm{d} S- \int_{ \Omega }  \nabla {v}\cdot  \nabla {u}\,\mathrm{d} x
        \]交换 \(  u,v  \)的地位  \[
        \int_{ \Omega }v \Delta u\,\mathrm{d} x= \int_{ \partial O}\frac{\partial u}{\partial \nu }v\,\mathrm{d} S-\int_{ \Omega }  \nabla _{u}\cdot  \nabla _{v}\,\mathrm{d} x
        \]两式相减即可.
    \end{enumerate}
    

    \hfill $\square$
\end{proof}

\begin{theorem}{散度算子的乘积律}
    \[
     \Delta \left( uv \right)= u \Delta v+ v \Delta u+ 2 \nabla u\cdot  \nabla v 
    \]
\end{theorem}

\section{极坐标}

\begin{theorem}{球坐标下的Laplace}
    考虑\(  \mathbb{R} ^{3}  \)上的球坐标 \(  \left( r, \theta , \varphi  \right)   \)  其上的Laplace算子表示为 \[
     \Delta f= \frac{1 }{r^{2} }\frac{\partial }{\partial r}\left( r^{2}\frac{\partial f}{\partial r} \right)+ \frac{1 }{r^{2}\sin  \theta  }\frac{\partial }{\partial  \theta }\left( \sin  \theta \frac{\partial f}{\partial  \theta } \right)+ \frac{1 }{r^{2}\sin ^{2} \theta  }\frac{\partial ^{2}f}{\partial  \varphi ^{2}}     
    \]特别地,若 \(  f= f\left( r \right)   \)是只依赖于径向的函数,则 \[
     \Delta f= \frac{1 }{r^{2} }\frac{\partial }{\partial r} \left( r^{2}\frac{\partial f}{\partial r} \right)= \frac{2 }{r }\frac{\partial f}{\partial r}  + \frac{\partial ^{2}f}{\partial r^{2}}
    \] 
\end{theorem}

\end{document}