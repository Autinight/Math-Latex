\documentclass[../../main.tex]{subfiles}

\begin{document}

\chapter{ 特征函数法 }

\section{基本概念}

考虑一下形式的线性PDE: \[
\begin{aligned}
 &\mathrm{E}.\mathrm{Q}:\quad \partial _{t}u= L_{x}u+ F\left( x,t \right) ,\quad \text{or}\quad \frac{\partial ^{2}u}{\partial t^{2}}= L_{x}u+ F\left( x,t \right)   \\ 
  & \mathrm{B}.\mathrm{C}: \quad B[u\left( x,t \right) ]= g\left( x,t \right),\quad \text{on the boundary} \partial  \Omega \\ 
   & \mathrm{I}.\mathrm{C}:\quad  u\left( x,0 \right)= f\left( x \right)   , \quad \text{or} \quad  u\left( x,0 \right)= f\left( x \right),  \partial _{t}u\left( x,0 \right)= h\left( x \right)    
\end{aligned}
\]


\begin{definition}{特征值与特征函数}
    对于上述问题中的空间算子 \(  L_{x}  \),和区域 \(   \Omega   \).若存在值 \(   \lambda   \)和函数 \(  X\left( x \right)   \),满足 \[
    L_{x}X\left( x \right)+ \lambda X\left( x \right)=    0
    \]并且 \[
    B\left[ X\left( x \right)  \right]= 0 
    \]    则称 \(   \lambda   \)是区域上空间算子的一个特征值, \(  X\left( x \right)   \)是关于 \(   \lambda   \)的一个特征函数.   
\end{definition}

\begin{definition}{内积}
    在函数空间中,两个是指函数 \(  f\left( x \right),g\left( x \right)    \)在区域 \(   \Omega   \)上的标准 \(  L^{2}  \)内积定义为 \[
    \left<f,g \right>= \int_{ \Omega }f\left( x \right)g\left( x \right)\,\mathrm{d} V  
    \]   
\end{definition}
\begin{theorem}{Laplace算子特征函数的正交性}
    对于Laplace算子 \(   \Delta   \)在有界区域 \(   \Omega   \)上.满足齐次Dirichlet或齐次Neumann边界条件的特征函数,如果它们对应于不同的特征值,那么它们是正交的.  
\end{theorem}

\begin{proof}
    设 \[
     \Delta  \varphi _{m}=  \lambda _{m} \varphi _{m},\quad  \Delta  \varphi _{n}=  \lambda _{n} \varphi _{n},\quad  \lambda _{m}\neq  \lambda _{n}
    \]由Green第二恒等式, \[
    \int_{ \Omega }\left(  \varphi _{m} \Delta  \varphi _{n}- \varphi _{n} \Delta  \varphi _{m} \right)\,\mathrm{d} V= \int_{ \partial  \Omega }\left(  \varphi _{m}\frac{\partial  \varphi _{n}}{\partial n}- \varphi _{n}\frac{\partial  \varphi _{m}}{\partial n} \right)\,\mathrm{d} S  
    \]将特征方程带入左侧积分,化为 \[
    \left(  \lambda _{n}- \lambda _{m} \right)\left< \varphi _{m}, \varphi _{n} \right> 
    \]
    检查右侧积分:\begin{enumerate}
        \item Dirichlet: 若 \(   \varphi _{m}= 0, \varphi _{n}= 0  \)在 \(   \partial  \Omega   \)上成立,则右侧积分化为 \[
        \int_{ \partial  \Omega }\left( 0\cdot \frac{\partial  \varphi _{n}}{\partial n}-0\cdot \frac{\partial  \varphi _{m}}{\partial n} \right)\,\mathrm{d} S= 9 
        \]  
        \item Neumann: 若 \(  \frac{\partial  \varphi _{m}}{\partial n}= 0  \),\(  \frac{\partial  \varphi _{n}}{\partial n}= 0  \)在 \(   \partial  \Omega   \)上成立,右侧积分化为 \[
        \int_{ \partial  \Omega }\left(  \varphi _{m}\cdot 0- \varphi _{n}\cdot 0 \right)\,\mathrm{d} S= 0 
        \]   
    \end{enumerate}
    因此,在以上两种边界条件下,均有 \[
    \left(  \lambda _{n}- \lambda _{m} \right)\left< \varphi _{m}, \varphi _{n} \right>= 0 
    \]则当 \(   \lambda _{n}\neq  \lambda _{m}  \)时, \[
    \left< \varphi _{m}, \varphi _{n} \right>= 0
    \] 

    \hfill $\square$
\end{proof}

\begin{theorem}{完备性定理}
    对于一个有界区域 $\Omega \subset \mathbb{R}^n$,其边界 $\partial \Omega$ 足够光滑(例如 $C^1$ 或分段光滑),Laplace 算子 $\Delta$ 在上述任何一种齐次边界条件下,都拥有一系列离散的、正的特征值:
$$ 0 < \lambda_1 \le \lambda_2 \le \lambda_3 \le \dots $$
(对于 Neumann 边界条件,$\lambda_1=0$ 对应常数特征函数)。这些特征值趋于无穷大 ($\lambda_n \to \infty$ as $n \to \infty$),且每个特征值都有有限的重数(即对应有限个线性独立的特征函数)。

更重要的是,对应的特征函数集合 $\{\phi_n(x)\}_{n=1}^\infty$ 构成了一个完备的正交基 (Complete Orthonormal Basis) 在 $L^2(\Omega)$ 空间中(或者更精确地说,在满足相应齐次边界条件的 $L^2(\Omega)$ 的子空间中)。

这意味着,对于任何函数 $f(x) \in L^2(\Omega)$(且满足相应的齐次边界条件),它可以被唯一地表示为这些特征函数的无限线性组合:
$$ f(x) = \sum_{n=1}^{\infty} c_n \phi_n(x) $$
其中系数 $c_n$ 可以通过内积(利用正交性)计算得到:
$$ c_n = \frac{\langle f, \phi_n \rangle}{\langle \phi_n, \phi_n \rangle} = \frac{\int_{\Omega} f(x) \overline{\phi_n(x)} dx}{\int_{\Omega} |\phi_n(x)|^2 dx} $$
(如果特征函数已经归一化,即 $\int_{\Omega} |\phi_n(x)|^2 dx = 1$,则分母为 1)。

这个级数在 $L^2$ 范数下收敛,即:
$$ \lim_{N \to \infty} \left\| f - \sum_{n=1}^{N} c_n \phi_n \right\|_{L^2} = \lim_{N \to \infty} \left( \int_{\Omega} \left| f(x) - \sum_{n=1}^{N} c_n \phi_n(x) \right|^2 dx \right)^{1/2} = 0 $$
\end{theorem}


之后均假设\(  L_{x}  \)是像 \(   \Delta   \)这样,具有可数个实   特征值,且特征函数有类似的完备性和正交性的算子. 
\section{齐次边界问题的解法}
\begin{theorem}
    对于\(  L_{x}, \Omega   \).设 \(  \left\{  \lambda _{n} \right\}  \)是 \(  L_{x}  \)的所有特征值, \(  \left\{ X_{n} \right\}  \)是其对应的一族特征函数. 则解 \(  u\left( x,t \right)   \)和非齐次源项 \(  F\left( x,t \right)   \)可以按特征函数展开: \[
    u\left( x,t \right)= \sum _{n = 1}^{\infty} T_{n}\left( t \right)X_{n}\left( x \right),\quad F\left( x,t \right)= \sum _{n = 1}^{\infty}F_{n}\left( t \right)X_{n}\left( x \right)      
    \]    其中   \[
    F_{n}\left( t \right)= \frac{\left<F\left( x,t \right),X_{n}\left( x \right)   \right> }{\left<X_{n}\left( x \right),X_{n}\left( x \right)   \right> }  
    \]且每个 \(  T_{n} \left( t \right)  \)都是形如下的ODE的解:
    \begin{enumerate}
        \item 对于一阶时间导数(热方程): \[
        T_{n}^{\prime} \left( t \right)+  \lambda _{n}T_{n}\left( t \right)= F_{n}\left( t \right)   
        \]
        \item 对于二阶时间导数(波动方程): \[
        T_{n}^{\prime \prime} \left( t \right)+  \lambda _{n}T_{n}\left( t \right)  = F_{n}\left( t \right) 
        \]
    \end{enumerate}
    
\end{theorem}
\begin{proof}
    由于 \(  \left\{ X_{n} \right\}  \)构成一族完备基,固定 \(  t  \), \(  u\left( x,t \right)   \)总可以按 \(  X_{n}\left( x \right)   \)展开,关于每个 \(  t  \)的系数函数即为 \(  T_{n}\left( t \right)   \)  .    
    以一阶时间导数的方程为例
    将展开式带入原始PDE,得到  \[
    \sum _{n = 1}^{\infty}T_{n}^{\prime} \left( t \right)X_{n}\left( x \right)= L_{x}\left( \sum _{n = 1}^{\infty}T_{n}\left( t \right)X_{n}\left( x \right)   \right)+ \sum _{n = 1}^{\infty}F_{n}\left( t \right)X_{n}\left( x \right)     
    \]由于 \(  L_{x}  \)是线性的,且 \(  T_{n}\left( t \right)   \)不依赖于 \(  x  \),带入特征关系,得到 \[
    \sum _{n = 1}^{\infty}T_{n}^{\prime} \left( t \right)X_{n}\left( x \right)  =- \sum _{n = 1}^{\infty} \lambda _{n}T_{n}\left( t \right)X_{n}\left( x \right)+ \sum _{n = 1}^{\infty}F_{n}\left( t \right)X_{n}\left( x \right)    
    \]   即 \[
    \sum _{n = 1}^{\infty}\left( T_{n}^{\prime} \left( t \right)+ \lambda _{n}T_{n}\left( t \right)-F_{n}\left( t \right)    \right)X_{n}\left( x \right)= 0  
    \]两边与 \(  X_{n}  \)做内积,由正交性得到 \[
    T_{n}^{\prime} \left( t \right)+  \lambda _{n}T_{n}\left( t \right)-F_{n}\left( t \right)= 0   
    \] 

    \hfill $\square$
\end{proof}
\begin{theorem}
    承接上面的定理.除了上面这些之外,初始条件 \(  u\left( x,t \right)= f\left( x \right)    \)展开为: \[
    f\left( x \right)= \sum _{n = 1}^{\infty}B_{n}X_{n}\left( x \right)  ,\quad B_{n}= \frac{\left<f,X_{n} \right> }{\left<X_{n},X_{n} \right> } 
    \]  \[
    h\left( x \right)= \sum _{n = 1}^{\infty}H_{n}X_{n}\left( x \right),\quad H_{n}= \frac{\left<h,X_{n} \right> }{\left<X_{n},X_{n} \right> }   
    \]
    \begin{enumerate}
        \item 一阶时间导数: 方程 \[
        T_{n}^{\prime} \left( t \right)+ \lambda _{n}T_{n}\left( t \right)= F_{n}\left( t \right)   
        \]具有初始条件 \(  T_{n}\left( 0 \right)= B_{n}   \) 确定出唯一解.
        \item 二阶时间导数: 方程 \[
        T_{n}^{\prime \prime} \left( t \right)+ \lambda _{n}T_{n}\left( t \right)= F_{n}\left( t \right)   
        \]具有初始条件 \(  T_{n}\left( 0 \right)= B_{n}   \), \(  T^{\prime} _{n}\left( 0 \right)= H_{n}   \)  ,确定出唯一解.
    \end{enumerate}
    
\end{theorem}

\section{处理非齐次边界条件}

\begin{theorem}
    若 \(  w\left( x,t \right)   \)满足非齐次边界条件,且 \(  L_{x}w  \)尽可能简单. 令 \[
    \tilde{F}\left( x,t \right)= F\left( x,t \right)+ L_{x}w-\frac{\partial w}{\partial t}  
    \]并令 \(  v\left( x,t \right)   \)是以下齐次边界问题的解  \[
\begin{aligned}
 &\mathrm{E}.\mathrm{Q}:\quad \partial _{t}v= L_{x}v+ \tilde{F}\left( x,t \right) ,\quad \text{or}\quad \frac{\partial ^{2}v}{\partial t^{2}}= L_{x}v+ F\left( x,t \right)   \\ 
  & \mathrm{B}.\mathrm{C}: \quad B[v\left( x,t \right) ]= 0,\quad \text{on the boundary} \partial  \Omega \\ 
   & \mathrm{I}.\mathrm{C}:\quad  v\left( x,0 \right)= f\left( x \right)   , \quad \text{or} \quad  v\left( x,0 \right)= f\left( x \right),  \partial _{t}v\left( x,0 \right)= h\left( x \right)    
\end{aligned}
\]令 \[
u\left( x,t \right)= w\left( x,t \right)+ v\left( x,t \right)   
\]则 \(  u  \)是原问题的解. 
\end{theorem}
\begin{proof}
    \[
    B[u]= B[w]+ B[v]= g\left( x,t \right) 
    \]
     \[
   \begin{aligned}
      \partial _{t}u&=  \partial _{t}w+  \partial _{t}v\\ 
       &=  \partial _{t}w+ L_{x}v+ \tilde{F}\left( x,t \right)\\ 
        &=  \partial _{t}w+ L_{x}u-L_{x}w + F\left( x,t \right)+ L_{x}w- \partial _{t}w\\ 
         &= L_{x}u+ F\left( x,t \right)  
   \end{aligned}
     \]
    \hfill $\square$
\end{proof}

\begin{problemsec}
    
\end{problemsec}
\begin{problem}
    求解以为波动方程的初边值问题: \[
    \begin{cases}  \partial _{tt}u-  \partial _{x x }u= \sin x,&0< x< \pi ,t> 0\\ 
     u\left( 0,t \right)= u\left( \pi ,t \right)= 0,&t> 0\\ 
      u\left( x,0 \right)= \frac{1 }{2 }\sin 2x, \partial _{t}u\left( x,0 \right)= 0,&0\le x\le \pi       \end{cases} 
    \]
\end{problem}

\begin{proof}
    
    方程为 \[
     \partial _{tt}u=  \partial _{xx }u+ \sin x
    \]考虑特征值问题 \[
    \begin{cases}  \partial _{x x}X+  \lambda X=  0,\\ 
     B[X]= 0 \end{cases} 
    \]\(   \partial _{x x}=  \Delta _{x}  \)有非负的特征值, 解得 \[
    X\left( x \right) = C_{ \lambda }\cos \left( \sqrt{ \lambda }x \right)  + D_{ \lambda }\sin \left(  \sqrt{ \lambda }x\right) 
    \]带入边界条件,得到 \[
    C_{ \lambda }= 0,\quad D_{ \lambda }\sin \left( \sqrt{ \lambda }\pi  \right)= 0 
    \]特征值 \(   \lambda   \)满足 \[
    \sqrt{ \lambda }\pi = n\pi ,\quad n = 1,2,\cdots 
    \]  即 \[
     \lambda _{n}= n^{2}
    \]取 \(  X_{n}\left( x \right)= \sin \left( nx \right)    \) .则 \(  u\left( x,t \right)   \)展开为 \[
    u\left( x,t \right)= \sum _{n = 1}^{\infty}T_{n}\left( t \right) X_{n}\left( x \right)  
    \] 其中, \(  T_{n}\left( t \right)   \)满足方程 \[
    T_1^{\prime \prime} \left( t \right)+ T_1\left( t \right)= 1,\quad T_{n}^{\prime \prime} \left( t \right)+ n^{2}T_{n}\left( t \right)= 0,n\ge 2    
    \] 解得  \[
    T_1\left( t \right)= A_1\cos \left( t \right)+ B_1\sin \left( t \right)+ 1   
    \] \[
    T_{n}\left( t \right)= A_{n}\cos \left( nt \right)+ B_{n}\sin \left( nt \right)   
    \]带入初值, \[
   B_{n}=  T_{n}^{\prime} \left( 0 \right)=  0, n\ge 1
    \] \[
    A_{n}= T_{n}\left( 0 \right)=  0,\quad n \neq 1, 2,\quad A_1= T_{1}\left( 0 \right)-1=- 1,\quad  A_2= T_2\left( 0 \right)= \frac{1}{2} 
    \]于是 \[
    T_1\left( t \right)=  -\cos t+ 1,\quad T_{2}\left( t \right) = \frac{1}{2}\cos \left( 2t \right),\quad T_{n}\left( t \right)= 0,\quad n \neq 1,2  
    \]最终 \[
    u\left( x,t \right)= \left( 1-\cos t \right)\sin \left( x \right)+ \frac{1}{2}\cos \left( 2t \right)    \sin \left( 2x \right) 
    \]

    \hfill $\square$
\end{proof}
\end{document}