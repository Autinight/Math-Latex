\documentclass[../../main.tex]{subfiles}

\begin{document}

\chapter{ Green函数 }



Green函数是在线性系统 \(  L  \)的作用下表现为“瞬时脉冲”的函数. 

\begin{definition}
    设\(   \Omega \subseteq \mathbb{R} ^{n}  \)是具有光滑边界的区域, \(  L  \)是一个线性微分算子, 定义关于以下非齐次方程 \[
    Lu\left( x \right)= f\left( x \right)  ,\quad x\in  \Omega 
    \] 的格林函数 \(  G\left( x,x^{\prime}  \right)   \)为以下分布意义下方程的解 \[
    LG\left( x,x^{\prime}  \right)=  \delta \left( x-x^{\prime}  \right)  ,\quad x,x^{\prime} \in  \Omega 
    \] 其中 \(  u\left( x \right)   \)是未知函数, \(   \delta   \)是Dirac分布.  
\end{definition}

\begin{theorem}\label{6.7-1}
   \(   \Omega ,L \)同前, \(  G  \)是方程 \[
   Lu\left( x \right)= f\left( x \right),\quad x\in  \Omega   
   \]的Green函数, 则 \[
   u\left( x \right)= \int_{ \Omega }G\left( x,x^{\prime}  \right)f\left( x^{\prime}  \right)  \,\mathrm{d} x^{\prime}  
   \]   是该方程的一个解.
\end{theorem}

\begin{proof}
    设 \(  u\left( x \right)   \)有题述表示, 由线性叠加原理, \[
    \begin{aligned}
    Lu\left( x \right)&= L \left( \int_{ \Omega }G\left( x,x^{\prime}  \right)f\left( x^{\prime}  \right)   \,\mathrm{d} x^{\prime}\right) \\ 
     &=  \int_{ \Omega } \left( LG\left( x,x^{\prime}  \right)  f\left( x^{\prime}  \right) \right)\,\mathrm{d} x^{\prime} \\ 
      &=  \int_{ \Omega }\left(  \delta \left( x-x^{\prime}  \right)  \right)f\left( x^{\prime}  \right)\,\mathrm{d} x^{\prime} \\ 
       &= f\left( x \right)   
    \end{aligned}  
    \]

    \hfill $\square$
\end{proof}

\section{Dirichlet问题的Green函数法}

本节希望通过Green函数法,给出Dirichlet问题 \[
\begin{cases}  \Delta u\left( x \right)+ f\left( x \right)= 0,& \forall x\in  \Omega \\ 
 u\left( x \right)= g\left( x \right),&\forall x\in  \partial  \Omega      \end{cases} 
\]的解的表示形式.

\begin{definition}{Green of Dirichlet}
    定义区域 \(   \Omega   \)上 Dirichelet问题对应的Green函数,为以下问题的解 \[
    \begin{cases}  \Delta _{x}G\left( x,x^{\prime}  \right)=  \delta \left( x-x^{\prime}  \right),&\forall x\in  \Omega \\ 
     G\left( x,x^{\prime}  \right)= 0,&\forall x\in  \partial  \Omega     \end{cases} 
    \]
\end{definition}


\subsection{非齐次方程,齐次边界}


\begin{theorem}
    \(   \Omega   ,G\left( x,x^{\prime}  \right) \)同前.令 \[
    u\left( x \right)= -\int_{ \Omega }G\left( x,x^{\prime}  \right)f\left( x^{\prime}  \right)\,\mathrm{d} x^{\prime}    
    \]则 \(  u\left( x \right)   \)是以下 有齐次边界条件的非齐次方程的解 \[
    \begin{cases}  \Delta u\left( x \right)+ f\left( x \right)= 0,&x\in  \Omega \\ 
     u\left( x \right)  = 0,&x\in  \partial  \Omega   \end{cases} 
    \]  
\end{theorem}

\begin{proof}
    结合定理\ref{6.7-1},这是显然的

    \hfill $\square$
\end{proof}

\subsection{齐次方程,非齐次边界}

\begin{theorem}
    \(   \Omega ,G  \)同前,令 \[
    u\left( x \right)= \int_{ \Omega }g\left( x^{\prime}  \right) \frac{\partial G\left( x^{\prime} ,x \right) }{\partial x^{\prime} }\,\mathrm{d} S_{x^{\prime} }  
    \]则\(  u\left( x \right)   \)是以下带有非齐次边界的齐次方程的解 \[
    \begin{cases}  \Delta u\left( x \right)= 0,&x\in  \Omega \\ 
     u\left( x \right)= g\left( x \right),&x\in  \partial  \Omega     \end{cases} 
    \]  
\end{theorem}

\begin{proof}
    由Green第二恒等式,我们有 \[
    \int_{ \Omega } G\left( x,x_0 \right)\Delta u\left( x \right)-u\left( x \right) \Delta _{x}G\left( x,x_0 \right)\,\mathrm{d} x=  \int_{ \Omega }\left( G\left( x,x_0 \right)\frac{\partial u}{\partial n_{x}}-u\left( x \right)\frac{\partial G}{\partial n_{x}} \right) \,\mathrm{d} S_{x}      
    \]其中 
    \begin{enumerate}
        \item  \[
    \int_{ \Omega } \Delta u\left( x \right)G\left( x,x_0 \right)\,\mathrm{d} x= \int_{ \Omega }0\cdot G\left( x,x_0 \right)\,\mathrm{d} x= 0   
    \] 
    \item \[
    \int_{ \Omega }u\left( x \right) \Delta _{x}G\left( x,x_0 \right)\,\mathrm{d} x=  \int_{ \Omega }u\left( x \right) \delta \left( x-x_0 \right)\,\mathrm{d} x= u\left( x_0 \right)     
    \]
    \item \[
    \int_{ \Omega }G\left( x,x_0 \right)\frac{\partial u}{\partial n_{x}}\,\mathrm{d} S_{x}= \int_{ \Omega }0\cdot \frac{\partial u}{\partial n_{x}} \,\mathrm{d} S_{x}= 0
    \] 
    \item \[
    \int_{ \Omega }u\left( x \right)\frac{\partial G}{\partial n_{x}}\,\mathrm{d} S_{x}= \int_{ \Omega }g\left( x \right)\frac{\partial G}{\partial n_{x}}  \,\mathrm{d} S_{x}
    \]
    \end{enumerate}
   全部带入,得到 \[
    u\left( x_0 \right)= \int_{ \Omega }g\left( x \right)\frac{\partial G}{\partial n_{x}}\,\mathrm{d} S_{x}  
    \]以\(  x  \)代\(  x_0  \), 以\(  x^{\prime}   \)代 \(  x  \),得到     \[
    u\left( x \right)= -\int_{ \Omega }g\left( x ^{\prime} \right) \frac{\partial G\left( x^{\prime} ,x \right) }{\partial n_{x^{\prime} }}\,\mathrm{d} S_{x^{\prime} }  
    \]

    \hfill $\square$
\end{proof}
\subsection{最终表示}
 \begin{theorem}
    \(   \Omega ,G  \)同前,令  \[
    u\left( x \right)= \int_{ \Omega }g\left( x^{\prime}  \right)\frac{\partial G\left( x^{\prime} ,x \right) }{\partial x^{\prime} }\,\mathrm{d} S_{x^{\prime} }  -\int_{ \Omega }g\left( x,x^{\prime}  \right)f\left( x^{\prime}  \right)\,\mathrm{d} x^{\prime}   
    \]则\(  u\left( x \right)   \)是以下带有非齐次边界的非齐次方程的解 \[
    \begin{cases}  \Delta u\left( x \right)= f\left( x \right),&x\in  \Omega \\ 
     u\left( x \right)= g\left( x \right),&x\in  \Omega      \end{cases} 
    \] 
 \end{theorem}
 \begin{proof}
    由线性叠加原理立即得到.
 
    \hfill $\square$
 \end{proof}

 \section{波动方程的Green函数}

我们考虑以下在有界区域 $\Omega \subset \mathbb{R}^n$ 上的非齐次波动方程初始-边界值问题 (IBVP):

$$ \begin{cases} \frac{\partial^2 u}{\partial t^2} - c^2 \Delta u = f(x,t) & \text{in } \Omega \times (0, T] \\ u(x,0) = g(x) & \text{on } \Omega \\ \frac{\partial u}{\partial t}(x,0) = h(x) & \text{on } \Omega \\ u(x,t) = \phi(x,t) & \text{on } \partial \Omega \times (0, T] \quad \text{(Dirichlet Boundary Condition)} \end{cases} $$
其中 $c>0$ 是波速,$f(x,t)$ 是源项,$g(x)$ 是初始位移,$h(x)$ 是初始速度,$\phi(x,t)$ 是边界上的位移。


\begin{definition}{波动方程的Green函数}
    
波动方程的Green函数 $G(x,t; x_0, t_0)$ 是满足以下条件的解:

$$ \begin{cases} \frac{\partial^2 G}{\partial t^2} - c^2 \Delta G = \delta(x - x_0) \delta(t - t_0) & \text{in } \Omega \times (0, T] \\ G(x,t; x_0, t_0) = 0 & \text{on } \partial \Omega \times (0, T] \\ G(x,t; x_0, t_0) = 0 & \text{for } t < t_0 \\ \frac{\partial G}{\partial t}(x,t; x_0, t_0) = 0 & \text{for } t < t_0 \end{cases} $$

其中:
\begin{itemize}
    \item $x \in \Omega$, $t \in (0, T]$ 是观测点和时间。
    \item $x_0 \in \Omega$, $t_0 \in (0, T]$ 是点源的位置和发生时间。
    \item 最后两个条件 $G(x,t; x_0, t_0) = 0$ for $t < t_0$ 和 $\frac{\partial G}{\partial t}(x,t; x_0, t_0) = 0$ for $t < t_0$ 确保了因果性,即波只在源出现之后才传播,并且在源出现之前,系统处于静止状态。

\end{itemize}

\end{definition}


\subsection{只有源项的贡献}

\begin{theorem}
    我们考虑以下齐次初始条件和齐次边界条件下的非齐次波动方程:
    $$ \begin{cases} \frac{\partial^2 u_f}{\partial t^2} - c^2 \Delta u_f = f(x,t) & \text{in } \Omega \times (0, T] \\ u_f(x,0) = 0 & \text{on } \Omega \\ \frac{\partial u_f}{\partial t}(x,0) = 0 & \text{on } \Omega \\ u_f(x,t) = 0 & \text{on } \partial \Omega \times (0, T] \end{cases} $$
    设 \(  G \left( x,t;x^{\prime} ,t^{\prime}  \right)  \) 是波动方程的Green函数,则 \[
    u_{f}\left( x,t \right)= \int_{0}^{t}\int_{ \Omega }G\left( x,t;x^{\prime} ,t^{\prime}  \right)f\left( x^{\prime} ,t^{\prime}  \right)\,\mathrm{d} x^{\prime} \,\mathrm{d} t^{\prime}    
    \]是问题的解.
\end{theorem}
\begin{proof}
    只验证源项,由算子的线性, \[
    \begin{aligned}
    \left(  \partial _{tt}-c^{2} \Delta  \right)u_{f}\left( x,t \right) &=  \int_{0}^{t}\int_{ \Omega } \left( \left(  \partial _{tt}-c^{2} \Delta  \right)G \right)  f\,\mathrm{d} x^{\prime} \,\mathrm{d} t^{\prime} \\ 
     &=    \int_{0}^{t}\int_{ \Omega } \delta \left( x-x^{\prime}  \right) \delta \left( t-t^{\prime}  \right)f\left( x^{\prime} ,t^{\prime}  \right)\,\mathrm{d} x^{\prime} \,\mathrm{d} t^{\prime}\\ 
      &= \int_{0}^{t} \delta \left( t-t^{\prime}  \right)f\left( x,t^{\prime}  \right)\,\mathrm{d} t^{\prime}       \\ 
       &= f\left( x,t \right) 
    \end{aligned}
    \]

    \hfill $\square$
\end{proof}
\subsection{只有初始位移的贡献}
\begin{theorem}
    我们考虑以下齐次源项、齐次初始速度和齐次Dirichlet边界条件下的波动方程:
    $$ \begin{cases} \frac{\partial^2 u_g}{\partial t^2} - c^2 \Delta u_g = 0 & \text{in } \Omega \times (0, T] \\ u_g(x,0) = g(x) & \text{on } \Omega \\ \frac{\partial u_g}{\partial t}(x,0) = 0 & \text{on } \Omega \\ u_g(x,t) = 0 & \text{on } \partial \Omega \times (0, T] \end{cases} $$
    设 $G(x,t; x',t')$ 是波动方程的Green函数,则
    $$ u_{g}(x,t)= -\int_{ \Omega }g(x^{\prime} )\frac{\partial G}{\partial t^{\prime}}(x,t;x^{\prime} ,0)\,\mathrm{d} x^{\prime} $$
    是上述问题的解。
\end{theorem}
\begin{proof}
    只验证初始位移项, 延迟Green函数在 \(  t= t^{\prime}   \)处具有跳跃条件  \[
    \lim_{t \to t^{\prime} }\frac{\partial G}{\partial t}\left( x,t;x^{\prime} ,t^{\prime}  \right)=  \delta \left( x-x^{\prime}  \right)  
    \]将极限过程反向,由对称性 \[
    \lim_{t^{\prime} \to t}\frac{\partial G}{\partial t^{\prime} }\left( x,t;x^{\prime} ,t^{\prime}  \right)=  - \delta \left( x-x^{\prime}  \right)  
    \]于是\[
    \begin{aligned}
    u_{g}\left( x,0 \right)&= - \int_{ \Omega }g\left( x^{\prime}  \right)\frac{\partial G}{\partial t^{\prime} }\left( x,0;x^{\prime} ,0 \right)\,\mathrm{d} x^{\prime} \\ 
     &=     -\int_{ \Omega }g\left( x^{\prime}  \right) \delta \left( x-x^{\prime}  \right)  \,\mathrm{d} x^{\prime}    \\ 
      &= g\left( x \right)
    \end{aligned}
    \]

    \hfill $\square$
\end{proof}

\subsection{只有初速度的贡献}
\begin{theorem}
    我们考虑以下齐次源项、齐次初始位移和齐次Dirichlet边界条件下的波动方程:
    $$ \begin{cases} \frac{\partial^2 u_h}{\partial t^2} - c^2 \Delta u_h = 0 & \text{in } \Omega \times (0, T] \\ u_h(x,0) = 0 & \text{on } \Omega \\ \frac{\partial u_h}{\partial t}(x,0) = h(x) & \text{on } \Omega \\ u_h(x,t) = 0 & \text{on } \partial \Omega \times (0, T] \end{cases} $$
    设 $G(x,t; x',t')$ 是波动方程的Green函数,则
    $$ u_{h}(x,t)= \int_{ \Omega }h(x^{\prime} )G(x,t;x^{\prime} ,0)\,\mathrm{d} x^{\prime} $$
    是上述问题的解。
\end{theorem}
\begin{proof}
    只验证初始速度项, 延迟Green函数在 \(  t= t^{\prime}   \)处具有跳跃条件  \[
    \lim_{t \to t^{\prime} }\frac{\partial G}{\partial t}\left( x,t;x^{\prime} ,t^{\prime}  \right)=  \delta \left( x-x^{\prime}  \right)  
    \]\[
    \begin{aligned}
     \partial _{t}u_{h}\left( x,0 \right)&= \int_{ \Omega }h\left( x^{\prime}  \right) \delta \left( x-x^{\prime}  \right) \delta ^{\prime}\mathrm{d} x^{\prime} = h\left( x \right)    
    \end{aligned}   
    \]

    \hfill $\square$
\end{proof}
\subsection{只有边界条件的贡献}

\begin{theorem}
    我们考虑以下齐次源项和齐次初始条件下的非齐次Dirichlet边界条件波动方程:
    $$ \begin{cases} \frac{\partial^2 u_\phi}{\partial t^2} - c^2 \Delta u_\phi = 0 & \text{in } \Omega \times (0, T] \\ u_\phi(x,0) = 0 & \text{on } \Omega \\ \frac{\partial u_\phi}{\partial t}(x,0) = 0 & \text{on } \Omega \\ u_\phi(x,t) = \phi(x,t) & \text{on } \partial \Omega \times (0, T] \end{cases} $$
    设 $G(x,t; x',t')$ 是波动方程的Green函数,则
    $$ u_{\phi}(x,t)= -c^2 \int_{0}^{t}\int_{\partial \Omega }\phi(x^{\prime} ,t^{\prime} )\frac{\partial G}{\partial \nu^{\prime}}(x,t;x^{\prime} ,t^{\prime} )\,\mathrm{d} S(x^{\prime} )\,\mathrm{d} t^{\prime} $$
    是上述问题的解。
\end{theorem}
\subsection{最终表示}

\begin{theorem}
    我们考虑以下在有界区域 $\Omega \subset \mathbb{R}^n$ 上的非齐次波动方程初始-边界值问题 (IBVP):
    $$ \begin{cases} \frac{\partial^2 u}{\partial t^2} - c^2 \Delta u = f(x,t) & \text{in } \Omega \times (0, T] \\ u(x,0) = g(x) & \text{on } \Omega \\ \frac{\partial u}{\partial t}(x,0) = h(x) & \text{on } \Omega \\ u(x,t) = \phi(x,t) & \text{on } \partial \Omega \times (0, T] \quad \text{(Dirichlet Boundary Condition)} \end{cases} $$
    设 $G(x,t; x',t')$ 是波动方程的Green函数,则上述IBVP的解 $u(x,t)$ 可以表示为:
    $$ u(x,t) = \int_{0}^{t}\int_{ \Omega }G(x,t;x^{\prime} ,t^{\prime}  )f(x^{\prime} ,t^{\prime}  )\,\mathrm{d} x^{\prime} \,\mathrm{d} t^{\prime} $$
    $$ - \int_{ \Omega }g(x^{\prime} )\frac{\partial G}{\partial t^{\prime}}(x,t;x^{\prime} ,0)\,\mathrm{d} x^{\prime} $$
    $$ + \int_{ \Omega }h(x^{\prime} )G(x,t;x^{\prime} ,0)\,\mathrm{d} x^{\prime} $$
    $$ - c^2 \int_{0}^{t}\int_{\partial \Omega }\phi(x^{\prime} ,t^{\prime} )\frac{\partial G}{\partial \nu^{\prime}}(x,t;x^{\prime} ,t^{\prime} )\,\mathrm{d} S(x^{\prime} )\,\mathrm{d} t^{\prime} $$
    其中 $\frac{\partial G}{\partial \nu'}$ 表示Green函数对源空间变量 $x'$ 在边界 $\partial \Omega$ 上的外法向导数。
\end{theorem}


\subsection{波动方程的Green函数}


\begin{theorem}
    以下给出几个空间上波动方程的Green函数
    \begin{enumerate}
        \item 三维 \[
        G\left( x,t;x_0,t_0 \right)= \frac{1 }{4\pi rc^{2} } \delta \left( t-t_0-\frac{r }{c }  \right)   
        \]或者使用相对坐标表示 \[
        G\left( r,\tau  \right)= \frac{1 }{4\pi rc^{2} } \delta \left( \tau -\frac{r }{c }  \right)   
        \]
        \item 二维 \[
        G\left( x,t;x_0,t_0 \right)= \frac{1 }{2\pi c\sqrt{c^{2}\left( t-t_0 \right)^{2} }-r^{2} }H\left( t-t_0-\frac{r }{c }  \right)   
        \]或者使用相对坐标表示 \[
        G\left( r,\tau  \right)= \frac{H\left( r- \frac{r }{c }  \right)  }{2\pi c\sqrt{c^{2}\tau ^{2}-r^{2}} }  
        \]
        \item 一维 \[
        G\left( x,t;x_0,t_0 \right)= \frac{1 }{2c }H\left( c\left( t-t_0 \right)-\left| x-x_0 \right|   \right)   
        \]或者使用相对坐标表示 \[
        G\left( r,\tau  \right)= \frac{1 }{2c }H\left( c\tau -r \right)   
        \]
    \end{enumerate}
    
\end{theorem}
\section{热方程}

\subsection{热方程的Green函数}
\begin{theorem}
    \begin{enumerate}
        \item 三维热方程的Green函数为 \[
        G\left( x,t;x_0,t_0 \right)= \frac{1 }{\left[ 4\pi k\left( t-t_0 \right) \right] ^{\frac{3}{2}}  }e^{-\frac{\left| x-x_0 \right|^{2}  }{4k\left( t-t_0 \right)  } }H\left( t-t_0 \right)   
        \]或者写成相对坐标形式 \[
        G\left( f,\tau  \right)= \frac{1 }{\left( 4\pi k\tau  \right)^{\frac{3}{2}}  }e^{-\frac{\left| r \right|^{2}  }{4 k\tau  } }H\left( \tau  \right)   
        \]
        \item 一维热方程的Green函数为 \[
        G\left( r,\tau  \right)= \frac{1 }{\sqrt{4\pi k\tau } }e^{-\frac{r^{2} }{4k \tau  } }H\left( \tau  \right)   
        \]
    \end{enumerate}
    
\end{theorem}

 \section{几种空间Green函数}

 \begin{theorem}
    \(   \Omega = \mathbb{R} ^{3}  \)上的拉普拉斯算子的Green函数为 \[
    G\left( x,x^{\prime}  \right)= -\frac{1 }{4\pi \left| x-x^{\prime}  \right|  }  
    \] 即上面的表达式在分布意义下满足 \[
    \begin{cases}  \Delta_{x}G\left( x,x^{\prime}  \right)=  \delta \left( x-x^{\prime}  \right) ,&x,x^{\prime} \in \mathbb{R} ^{3}\\ 
     \lim_{\left| x  \right|\to \infty }G\left( x,x^{\prime}  \right)= 0,&x^{\prime} \in \mathbb{R} ^{3}    \end{cases} 
    \]
 \end{theorem}

 \begin{proof}
    我们希望找到 \(  G\left( x,x^{\prime}  \right)   \),使得 \[
     \Delta _{x}G\left( x,x^{\prime}  \right)=  \delta \left( x-x^{\prime}  \right)  
    \] 方便起见,固定 \(  x^{\prime}   \),令 \(  y= x-x^{\prime}   \),则 \(   \Delta _{x}=  \Delta _{y}  \), 记 \(  G\left( y \right)= G\left( x,x^{\prime}  \right)    \),只需要找到 \(  G\left( y \right)   \),使得 \[
     \Delta _{y}G\left( y \right)=  \delta \left( y \right)  
    \]   希望寻找径向的 \(  G  \),  利用  \[
     \Delta f= \frac{1 }{r^{2} }\frac{\partial }{\partial r} \left( r^{2}\frac{\partial f}{\partial r} \right)= \frac{2 }{r }\frac{\partial f}{\partial r}  + \frac{\partial ^{2}f}{\partial r^{2}}
    \] 在 \(  y \neq 0  \)处,解方程 \[
    \frac{\partial G}{\partial r}\left( r^{2}\frac{\partial G}{\partial r} \right)= 0 
    \] 解得 \[
    G\left( r \right)= -\frac{C_1 }{r }+   C_2
    \]
    接下来确定 \(  C_1,C_2  \),任取测试函数 \(   \varphi \in \mathcal{D}\left( \mathbb{R} ^{3} \right)   \),  我们需要 \[
    \left< \Delta G, \varphi  \right>= \left< \delta , \varphi  \right>
    \]根据分布的导数的定义,以及Dirac函数的筛选性,上面写作 \[
    \left<G, \Delta  \varphi  \right>=  \varphi \left( 0 \right) 
    \]即 \[
    \int_{\mathbb{R} ^{3}}\left( -\frac{C_1 }{\left| x \right|  }+ C_2  \right) \Delta  \varphi \left( x \right)\,\mathrm{d} x=  \varphi \left( 0 \right)   
    \]由散度定理 \[
    \int_{\mathbb{R} ^{3}}C_2 \Delta  \varphi \left( x \right)\,\mathrm{d} x=  C_2 \int_{ \partial \mathbb{R} ^{3}} \nabla  \varphi \cdot \mathbf{n}\,\mathrm{d} x= 0
    \]其中最后的等号是因为 \(   \varphi   \)的紧支性导致的无穷远处的消失性. 我们发现 \(  C_2  \)不影响 \(   \Delta   G\)与 \(   \delta   \)的关系,通常取 \(  C_2= 0  \).    
    对于 \[
    \int_{\mathbb{R} ^{3}}\frac{1 }{\left| x \right|  } \Delta  \varphi \,\mathrm{d} x 
    \]在 \(  V_{ \varepsilon }: =  \mathbb{R} ^{3}\setminus B_{ \varepsilon }  \)上,使用第二格林公式,得到 \[
    \int_{V_{ \varepsilon }}\frac{1 }{\left| x \right|  } \Delta  \varphi \,\mathrm{d} x=\int_{V_{ \varepsilon }} \varphi  \Delta \left( \frac{1 }{\left| x \right|  }  \right)\,\mathrm{d} x-\int_{ \partial V_{ \varepsilon }}\left( \frac{1 }{\left| x \right|  }   \nabla  \varphi -  \varphi  \nabla \left( \frac{1 }{\left| x \right|  }  \right) \right)\cdot e_{r}\,\mathrm{d} S  
    \] 由于 \(   \varphi   \)是测试函数, \(  \varphi \)和 \(   \nabla  \varphi   \) 在无穷远处消失,并且 \(   \Delta \left( \frac{1 }{\left| x \right|  }  \right)= 0   \)在 \(  V_{ \varepsilon }  \)上成立 , 于是右侧积分化为  \[
\begin{aligned}
    &- \int_{ \partial _{ \varepsilon }}\left( \frac{1 }{\left| x \right|  } \nabla  \varphi +  \varphi  \nabla \left( \frac{1 }{\left| x \right|  }  \right)  \right)\cdot e_{r}\,\mathrm{d} S  \\ 
     &=- \int_{ \partial B_{ \varepsilon }}\left( \frac{1 }{r }\frac{\partial  \varphi }{\partial r}+ \frac{1 }{r^{2} } \varphi    \right)\,\mathrm{d} S\\ 
      &= - \int_{ \partial B_{ \varepsilon }}\left( \frac{1 }{r }\left( \left. \frac{\partial  \varphi   }{\partial r} \right|_{0}+O\left( r \right) \right) +  \frac{1 }{r^{2} }\left(  \varphi \left( 0 \right)+ \left. \frac{\partial  \varphi }{\partial r} \right|_{0}r+ O\left( r^{2} \right)   \right)     \right)  \,\mathrm{d} S\\ 
       &= -\int_{ \partial B_{ \varepsilon }}\left( \frac{1 }{r^{2} } \varphi \left( 0 \right)+ \frac{2 }{r }\left. \frac{\partial  \varphi }{\partial r} \right|_{0}+ O\left( 1 \right)     \right)\,\mathrm{d} S \to- 4\pi  \varphi \left( 0 \right),\left(  \varepsilon \to 0 \right)  
\end{aligned}    \]于是 \[
C_1 \left( 4\pi  \varphi \left( 0 \right)  \right)=  \varphi \left( 0 \right)  
\]得到 \[
C_1= \frac{1 }{4\pi  } 
\]进而\[
G\left( r \right)= -\frac{1 }{4\pi  r}  
\]即 \[
G\left( x,x^{\prime}  \right)= -\frac{1 }{4\pi \left| x-x^{\prime}  \right|  }  
\]
    \hfill $\square$
 \end{proof}


\begin{problemsec}
    
\end{problemsec}
\begin{problem}
\begin{enumerate}
    \item 定义 $\Psi(x) := \frac{1}{|x|} \operatorname{exp}(-|x|)$, $0 \neq x \in \mathbb{R}^3$. 证明 $\Psi(x)$ 满足如下方程:
    \[ -\Delta \Psi(x) + \Psi(x) = 0, \quad 0 \neq x \in \mathbb{R}^3. \]
    \item 并以此 (仿照调和方程的 Green 函数法) 求解如下定解问题:
    \[ \begin{cases} -\Delta u + u = 0, & x \in \mathbb{R}^3_+ \\ u|_{\partial \mathbb{R}^3_+} = g. \end{cases} \]
\end{enumerate}
\end{problem}

\begin{proof}
 \begin{enumerate}
   \item 令 \(  r= \left| x \right|   \), 设 \(  G\left( r \right)= \Psi \left( x \right)= \frac{1 }{r }\exp \left( -r \right)      \) ,则对于径向函数 \(  G\left( r \right)   \),其关于 \(  x  \)的 Laplace算子满足 \[
    \Delta G\left( r \right)=  \frac{1 }{r^{2} } \partial _{r}\left( r ^{2}\frac{\partial G}{\partial r} \right)  
   \] 计算即可.
   \item 

   根据无限域上Dirichlet上的基本解,我们已经知道 在分布的意义下,\[
    \Delta _{x}\left( -\frac{1 }{4\pi \left| x \right|  }  \right) =  \delta \left( x \right) 
   \]利用Laplace算子的乘积法则 \[
    \Delta \left( uv \right)= u \Delta v+ v \Delta u+ 2 \nabla u \nabla v 
   \]那么 \[
   \begin{aligned}
    &\Delta _{x}\left( -\frac{1 }{4\pi \left| x \right|  }\exp \left( -\left| x \right| \right)   \right)\\ 
     &=  \Delta _{x}\left( -\frac{1 }{4\pi \left| x \right|  }  \right)\exp \left( -\left| x \right|  \right)-\frac{1 }{4\pi \left| x \right|  } \Delta _{x}\exp \left( -x \right)+2  \nabla \left( -\frac{1 }{4\pi \left| x \right|  }  \right)\cdot  \nabla \left( \exp \left( -\left| x \right|  \right)  \right)       \\ 
      &=  \delta \left( x \right)\exp \left( -\left| x \right|  \right)-\frac{1 }{4\pi \left| x \right|  }   \left( 1-\frac{2}{\left| x \right|  }  \right) \exp \left( -\left| x \right|  \right) + 2  \left( -\frac{1 }{4\pi  }\frac{x }{\left| x \right|^{3}  }   \right) \exp \left( -\left| x \right|  \right)\left( -\frac{x }{ \left| x \right| }  \right)  \\ 
       &=  \delta \left( x \right)\exp \left( -\left| x \right|  \right)-\frac{1 }{4\pi \left| x \right|  }\exp \left( -\left| x \right|  \right)    
   \end{aligned}
   \]
于是在分布的意义下
   \[
   \Delta _{x}\left( -\frac{1 }{4\pi \left| x \right|  }\exp \left( -\left| x \right| \right)   \right)-\left( -\frac{1 }{4\pi \left| x \right|  }\exp \left( -\left| x \right|  \right)   \right)=  \delta \left( x \right)  \exp \left( -\left| x \right|  \right)=  \delta \left( x \right)  
   \]令 \(  G_0\left( x,x_0 \right)   = \frac{1 }{4\pi \left| x-x_0 \right|  }\exp \left( -\left| x-x_0 \right|  \right)  \) ,则下述方程在分布的意义下成立 \[
 \left( - \Delta _{x}+ I \right) G_0\left( x,x_0 \right)=   \delta \left( x-x_0 \right)  
   \]接下来,设 \(  x^{\prime}   \)是 \(  x  \)关于 \(   \partial \mathbb{R} _{+ }^{3}  \)的镜像对称点,定义 \[
   G\left( x,x_0 \right)= G_0\left( x,x_0 \right)-G_0\left( x^{\prime} ,x_0 \right)   
   \]   则由线性叠加原理,下述方程在分布意义下成立 \[
   \begin{cases} \left( - \Delta _{x}+ I \right)G\left( x,x_0 \right)=  \delta \left( x-x_0 \right),\quad x\in \mathbb{R} _{+ }^{3}\\ 
    G\left( x,x_0 \right)= 0,\quad x\in  \partial\mathbb{R}  _{+ }^{3}     \end{cases} 
   \]

   记 \(  L_{x}= \left( - \Delta _{x}+ I \right)   \)是一个线性微分算子,根据Green第二恒等式 \[
   \int_{\mathbb{R} _{+ }^{3}}u\left( x \right) \Delta _{x}G\left( x,x_0 \right)-G\left( x,x_0 \right) \Delta u\left( x \right)\,\mathrm{d} x= \int_{ \partial \mathbb{R} _{+ }^{3}}u\frac{\partial G}{\partial n_{x}}    -G\frac{\partial u}{\partial n_{x}} \,\mathrm{d} S
   \] 得到 \[
   \int_{\mathbb{R} _{+ }^{3}}G\left( x,x_0 \right)L_{x}  u\left( x \right)- u\left( x \right)L_{x}G\left( x,x_0 \right)  \,\mathrm{d} x= \int_{ \partial \mathbb{R} _{+ }^{3}} u\frac{\partial G}{\partial n_{x}}-G\frac{\partial u}{\partial n_{x}}\,\mathrm{d} S
   \]若 上面的 \(  u  \)满足 \[
   \begin{cases} L_{x}u= 0,&x\in \mathbb{R} _{+ }^{3}\\ 
    u|_{ \partial \mathbb{R} _{+ }^{3}}= g \end{cases} 
   \]  则上述积分式化为 \[
   -\int_{\mathbb{R} _{+ }^{3}}u\left( x \right) \delta \left( x-x_0 \right)\,\mathrm{d} x=  \int_{ \partial \mathbb{R} _{+ }^{3}}g\left( x \right)\frac{\partial G\left( x,x_0 \right) }{\partial n_{x}}\,\mathrm{d} S   
   \]其中左侧为 \(  u\left( x_0 \right)   \).依据此,取 \[
   u\left( x_0 \right)= -\int_{ \partial \mathbb{R} _{+ }^{3}}g\left( x \right)\frac{\partial G\left( x,x_0 \right) }{\partial n_{x}}\,\mathrm{d} S  
   \] 即 \[
   u\left( x \right)= -\int_{ \partial \mathbb{R} _{+ }^{3}} g\left( x_0 \right)\frac{\partial G\left( x_0,x \right) }{\partial n_{x_0}}\,\mathrm{d} S_{x_0} 
   \]带入回上述过程,可知 \(  u\left( x \right)   \)满足边界条件 \[
   u|_{ \partial \mathbb{R} _{+ }^{3}}= g
   \]此外,由微分算子\(  L  \) 的线性 \[
   L_{x}u= -\int_{ \partial \mathbb{R} _{+ }^{3}} g\left( x \right) \frac{\partial L_{x}G\left( x_0,x \right) }{\partial n_{x_0} }\,\mathrm{d} S_{x_0}= -\int_{ \partial \mathbb{R} _{+ }^{3}}g\left( x \right)\cdot 0= 0  
   \] 这上述构造的\(  u  \)确实是方程的解.
   
   最后,计算 \(  \frac{\partial G\left( x_0,x \right) }{\partial n_{x_0}}\),无非是 \(  x_0  \)关于第三个分量的偏导数的相反数,计算过程略去. 
 \end{enumerate}
 

    \hfill $\square$
\end{proof}
\end{document}