\documentclass[lang=cn,12pt,color=green,fontset=none]{elegantbook}

\title{数学分析}
\subtitle{詹伟城}

\author{Autin}

\setmainfont{Aa顺風顺水顺财神}
\setCJKmainfont{Aa顺風顺水顺财神}
\setCJKsansfont{Aa顺風顺水顺财神}
\setCJKmonofont{Aa顺風顺水顺财神}

%\extrainfo{不要以为抹消过去,重新来过,即可发生什么改变.—— 比企谷八幡}

\setcounter{tocdepth}{3}


\cover{image.png}
\usepackage{CJKutf8}
% 本文档命令
\usepackage{array}
\newcommand{\ccr}[1]{\makecell{{\color{#1}\rule{1cm}{1cm}}}}

% 修改标题页的橙色带
% \definecolor{customcolor}{RGB}{32,178,170}
% \colorlet{coverlinecolor}{customcolor}

\begin{document}

\maketitle
\frontmatter

\tableofcontents

\mainmatter

\chapter{多元积分}


\section{Green公式}

\begin{theorem}{面积公式}
     $$
     \sigma\left( D \right) = -\oint_{\partial D    } y \,\mathrm{d}x = \oint_{ \partial D} x \,\mathrm{d}y = \frac{1}{2} \oint_{\partial D} \left( -y \right)\,\mathrm{d}x +  x \,\mathrm{d}y 
     $$
\end{theorem}

\begin{proposition}{第二型曲线积分与第一型曲线积分的关系}
    $$
    \iint_{D}\left( - \frac{\partial P}{\partial y} +  \frac{\partial Q}{\partial x}  \,\mathrm{d}x \mathrm{d}y \right) = \oint_{\partial D} P \,\mathrm{d}x+  Q \,\mathrm{d}y= \int_{\partial D} \left( -P \vec{n_{y}} +  Q \vec{n_{x}}\right)\,\mathrm{d}s  
    $$其中 $  \vec{n}= \left(  \vec{n_{x}},\vec{n_{y}} \right)  $是 $ \partial D $的单位外法向量  
\end{proposition} 

\begin{corollary}
    
    设 $  D $同上, $ \vec{n} =\left( \vec{n_{x}  },\vec{n_{y}} \right)  $为 $ \partial D $   的单位外法向量, $  P,Q \in  C^{1}\left(  \overline{D} \right)  $则 $$
    \iint_{D} \left(  \frac{\partial P}{\partial x}+  \frac{\partial Q}{\partial y} \right) \,\mathrm{d}x \mathrm{d} y= \int_{ \partial D} \left(  P \vec{n }_{x} +  Q \vec{n}_{y} \right)  
    $$ 
\end{corollary}

\begin{corollary}{二维分部积分公式}
     $$
      \iint_{D} \frac{\partial P}{\partial x} \cdot Q \,\mathrm{d}x \mathrm{d}y = - \iint_{D} P \cdot  \frac{\partial Q}{\partial x} \,\mathrm{d}x \mathrm{d}y +  \int_{ \partial D} PQ \vec{n  }_{x} ds
     $$ $$
     \iint_{D} \frac{\partial P}{\partial y}\cdot Q \,\mathrm{d}x \mathrm{d} y = - \iint_{D} P \cdot \frac{\partial Q}{\partial y} \,\mathrm{d}x  \mathrm{d}y +  \int_{\partial D} PQ \vec{n}_{y} \,\mathrm{d}s
     $$
\end{corollary}

\begin{proof}
    以第一个式子为例 $$
    \begin{aligned}
       & \frac{\partial \left( PQ \right) }{\partial x} = \frac{\partial P}{\partial x} \cdot  Q+  P \frac{\partial Q}{\partial x} \\ 
         & \implies \iint_{D} \frac{\partial PQ}{\partial x} \,\mathrm{d}x \mathrm{d} y = \iint_{D} \frac{\partial P}{\partial x}Q \,\mathrm{d}x \,\mathrm{d}y +  \iint_{D} P \frac{\partial Q}{\partial x} \,\mathrm{d}x   \,\mathrm{d}y 
    \end{aligned}
    $$
    其中 $$
     \iint_{D} \frac{\partial PQ}{\partial x} \,\mathrm{d}x \mathrm{d} y = \int_{ \partial D        } PQ \vec{n_{x }} \,\mathrm{d}s
    $$ 
\end{proof}


\chapter{含参积分}

\section{Bochner-Lebesgue含参积分}

\begin{theorem}{连续性}
    设 $ M $是度量空间,$ f:X\times M\to E $满足 
    \begin{enumerate}
        \item $ f\left( \cdot ,m \right)\in \mathcal{L}_{1}\left( X,\mu ,E \right)   $对于每个 $ m \in M $成立;
        \item  $ f\left( x,\cdot  \right)\in C\left( M,E \right)   $对于 $ \mu  $-几乎所有 $ x \in X $成立;
        \item 存在 $ g \in \mathcal{L}_{1}\left( X,\mu ,E \right)  $,使得 $ \left| f\left( x,m \right)  \right|\le g\left( x \right)   $对于 $ \left( x,m \right)\in X\times M  $   成立.
    \end{enumerate}
    则 $$
    F:M\to E,\quad  m\mapsto \int_{X}f\left( x,m \right)\mu \left( \,\mathrm{d} x \right)  
    $$是良定义且连续的.
\end{theorem}
\begin{proof}
    良定义性由1.立即得到.设 $ m \in M $,令 $ \left( m_{j} \right)  $是在 $ M $中收敛到 $ m $的列.令 $ f_{j}: = f\left( \cdot ,m_{j} \right)  $, $ j \in \mathbb{N}  $,则由2., $ f_{j}\to f $ $ \mu  $-a.e.且由1, $ f_{j} \in \mathcal{L}_{1}\left( X,\mu ,E \right), j \in \mathbb{N}   $ ,由3. $ \left| f_{j} \right| \le  \left| g \right|  $   ,因此由控制收敛定理 $$
    \lim_{j \to \infty} F\left( m_{j} \right) = \lim_{j \to \infty}\int_{X}f_{j}=    \int_{X}\lim_{j \to \infty}f_{j} = \int_{X} f\left( x,m \right)\mu \left( \,\mathrm{d} x\right) = F\left( m \right)   
    $$这表明 $ F $是连续的. 

    \hfill $\square$
\end{proof}

\begin{theorem}{可微性}
    设 $ U $是 $ \mathbb{R} ^{n} $中的开集,设 $ f: X\times U\to E $满足 
    \begin{enumerate}
        \item $ f\left( \cdot ,y \right)\in \mathcal{L}_{1}\left( X,\mu ,E \right)   $对所有 $  y\in U $成立;
        \item  $ f\left( x,\cdot  \right) \in C^{1}\left( U,E \right)   $对于 $ \mu  $-几乎所有 $ x \in X $成立;
        \item 存在 $ g \in  \mathcal{L}_{1}\left( X,\mu,\mathbb{R}  \right)  $,使得 $$
        \left| \frac{\partial }{\partial y^{j}} f\left( x,Y \right) \right| \le g\left( x \right),\quad \left( x,y \right)\in X\times U,\quad 1\le j\le n  
        $$      
    \end{enumerate}
       则 $$
       F:U\to E,\quad y\mapsto \int_{X}f\left( x,y \right) \mu \left( \,\mathrm{d} x \right)  
       $$是连续可微的,并且 $$
       \partial _{j}F\left( y \right)= \int_{X} \frac{\partial }{\partial y^{j}} f\left( x,y \right) \mu \left( \,\mathrm{d} x \right),\quad  y \in U,\quad  1\le j\le n  
       $$
\end{theorem}
\end{document}
