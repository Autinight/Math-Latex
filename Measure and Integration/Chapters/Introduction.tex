\documentclass[../main.tex]{subfiles}

\begin{document}

\chapter{Introduction }
\section{Arithmetic in   [0,infty]    } 
\section{Metric Space}
\begin{lemma}{}{}
    \(  X  \) is a metric space, \(  x \in X  \), \(  B_1,B_2  \) are twos balls in \(  X  \). If \(  x  \in B_1\cap B_2  \) , then \(  x  \) is the center of an open ball \(  B\subseteq  B_1\cap B_2  \).       
\end{lemma}
\begin{proof}
    We may assume that \(  B_1,B_2  \) have  different center. Let  \(  y_1\neq y_2  \), \[
B_1= \left\{ x: d\left( x,y_1 \right)< r_1  \right\},\quad B_2= \left\{ x:d\left( x,y_2 \right)< r_2  \right\}
\] 
Take  \(  x \in B_1\cap B_2  \), we claim that there exists  \(  r_0> 0  \) such that  \(  B: =  \left\{ y: d\left( x,y \right)< r_0  \right\}\subseteq B_1\cap B_2  \). Otherwise , for all \(  r > 0 \),  there exists a point \(  y_{r}  \) such that   the following two hold at the same time \begin{enumerate}
    \item \(  d\left( x,y_{r} \right)< r   \)
    \item \(  d\left( y_{r},y_1 \right)\ge r_1   \) or \(  d\left( y_{r},y_2 \right)\ge r_2   \).   
\end{enumerate}
Thus \[
d\left( x,y_1 \right)\ge d\left( y_{r},y_1 \right)-d\left( x,y_{r} \right)>  r_1-r   ,\quad  \text{or} \quad  d\left( x,y_2 \right)> r_2-r 
\] The above shows that \[
r> \min \left\{ r_1-d\left( x,y_1 \right),r_2-d\left( x,y_2 \right)   \right\},\quad \forall r> 0
\] which is a contradiction.
\end{proof}

\section{Limits of Set Sequences}

\begin{definition}{Limit Inferior and Limit Superior}{}
    Let $\{A_n\}_{n=1}^\infty$ be a sequence of sets.
    \begin{enumerate}
        \item By the \dfntxt{Limit Inferior of the sequence}, we mean \[
        \liminf_{n \to \infty} A_n = \bigcup_{N=1}^\infty \bigcap_{n=N}^\infty A_n = \left\{ x:  \exists N_0\in \mathbb{N} , \forall n> N_0,x \in A_{n}\right\}
        \]
        \item By the \dfntxt{Limit Superior of the sequence},we mean \[
        \limsup_{n \to \infty} A_n = \bigcap_{N=1}^\infty \bigcup_{n=N}^\infty A_n =  \left\{ x: \forall N\in \mathbb{N} , \exists n\ge N, x\in A_{n} \right\}
        \]
    \end{enumerate}
    
\end{definition}
\begin{note}
    \begin{enumerate}
        \item Limit Inferior包含那些“最终稳定下来”的元素,即从某个点之后就永远属于序列中所有后续集合的元素。
        \item  Limit Superior 包含那些“反复出现”的元素,即在无限多个集合中出现的元素。
    \end{enumerate}

\end{note}

\begin{proposition}{}{}
    For any sequence of sets $\{A_n\}$, it always holds that:
$$\liminf_{n \to \infty} A_n \subseteq \limsup_{n \to \infty} A_n$$
\end{proposition}
\begin{proofsketch}
    直觉上是显然的,因为 “最终稳定下来”的元素一定会“反复出现”.
\end{proofsketch}

\begin{proof}
    It is obvious by the \(  \left\{ x:x \in P \right\}  \) form representation of the Limit Inferior and Superior.
\end{proof}

\begin{definition}{Limit of a Set Sequence}{}
    If the limit inferior and limit superior of a set sequence $\{A_n\}_{n=1}^\infty$ are equal, i.e., $\liminf_{n \to \infty} A_n = \limsup_{n \to \infty} A_n$, then we say the \dfntxt{limit} of the sequence exists, and it is defined as:
$$\lim_{n \to \infty} A_n = \liminf_{n \to \infty} A_n = \limsup_{n \to \infty} A_n$$
\end{definition}

\begin{proposition}{}{}
     Let $\{A_n\}_{n=1}^\infty$ be a sequence of sets.
     \begin{enumerate}
        \item If $\{A_n\}$ is an increasing sequence, then the limit of \(  \left\{ A_{n} \right\}   \) exists and \[
        \lim_{n\to \infty}A_{n}= \bigcup _{n = 1}^{\infty}A_{n}
        \]
        \item If \(  \left\{ A_{n} \right\}  \) is a decreasing sequence, then the limit of \(  \left\{ A_{n} \right\}   \) exists and  \[
        \lim_{n\to \infty}A_{n}= \bigcap_{n = 1}^{\infty}A_{n} 
        \] 
     \end{enumerate}
     
\end{proposition}
\begin{proof}
    \begin{enumerate}
        \item If \(  \left\{ A_{n} \right\}  \) is increasing , then  \[
        \bigcap_{n = N}^{\infty}A_{n}= A_{N} 
        \]and\[ 
       \bigcup _{n = N}^{\infty}A_{n}= \bigcup _{n =  1}^{\infty}A_{n},\forall N\in \mathbb{N} .
        \] Thus \[
        \liminf_{n \to \infty} A_n = \bigcup_{N=1}^\infty \bigcap_{n=N}^\infty A_n = \bigcup _{N = 1}^{\infty}A_{N}
        \] and\[
        \limsup_{n \to \infty} A_n = \bigcap_{N= 1}^{\infty}\bigcup _{n = N}^{\infty}A_{n}=  \bigcap_{N = 1}^{\infty} \bigcup _{n = 1}^{\infty}A_{n}=  \bigcup _{n = 1}^{\infty}A_{n}  
        \]which are the same.
        \item If \(  \left\{ A_{n} \right\}  \) is decreasing, then \[
        \bigcap_{n = N}^{\infty}A_{n}= \bigcap_{n = 1}^{\infty}A_{n}  
        \] and \[
        \bigcup _{n = N}^{\infty}A_{n}= A_{N}
        \]Thus  \[
        \liminf_{n \to \infty} A_n = \bigcup_{N=1}^\infty \bigcap_{n=N}^\infty A_n = \bigcup _{N = 1}^{\infty}\bigcap_{n = 1}^{\infty}A_{n}= \bigcap_{n = 1}^{\infty}A_{n}  
        \] and\[
        \limsup_{n \to \infty} A_n = \bigcap_{N= 1}^{\infty}\bigcup _{n = N}^{\infty}A_{n}= \bigcap_{N = 1}^{\infty}A_{N}   
        \]which are the same.
    \end{enumerate}
    
\end{proof}

\begin{proposition}{}{}
    For a sequence of sets $\{A_n\}_{n=1}^\infty$ and their corresponding sequence of indicator functions $\{\chi_{A_n}(x)\}_{n=1}^\infty$:
    \begin{itemize}
        \item 
$\chi_{\liminf_{n \to \infty} A_n}(x) = \liminf_{n \to \infty} \chi_{A_n}(x)$
\item $\chi_{\limsup_{n \to \infty} A_n}(x) = \limsup_{n \to \infty} \chi_{A_n}(x)$
\item  $\lim_{n \to \infty} A_n$ exists, then
$\chi_{\lim_{n \to \infty} A_n}(x) = \lim_{n \to \infty} \chi_{A_n}(x)$
    \end{itemize}
    
\end{proposition}
\begin{proofsketch}
    \begin{itemize}
        \item 若 \(  x  \)在limit inferior 里面,则\(  x  \)是“最终稳定”的, \(  \chi _{A_{n}}\left( x \right)   \)是关于 \(  n  \) “最终”恒为1的.
        \item 若 \(  x  \)在 limit superior里面,则 \(  x  \)是  “反复出现”的,即相当于 \(  N  \)多大,总会出现之后的某个 \(  n  \)使得 \(  \chi _{A_{n}}\left( x \right)= 1   \).
        \item 当极限存在时,函数列 \(  \left\{ \chi _{A_{n}}\left( x \right)  \right\}_{n = 1}^{\infty}  \)的limit inferior 和 supperior根据上两条相等,等于极限集合的 \(  \chi   \). 
    \end{itemize}
    
\end{proofsketch}

\end{document}